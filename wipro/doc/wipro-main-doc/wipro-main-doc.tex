\documentclass[a4paper, table]{article}

\usepackage{xcolor}
\usepackage{amssymb}
\usepackage[a4paper]{geometry}
\usepackage[T1]{fontenc}
\usepackage[utf8]{inputenc}
\usepackage{graphicx}
\usepackage{todonotes}
\usepackage{hyperref}
\usepackage{longtable}
\usepackage{times}
\usepackage{pdflscape}
\usepackage[style=verbose-ibid,backend=bibtex]{biblatex}
\usepackage{listings}

\definecolor{dkgreen}{rgb}{0,0.6,0}
\definecolor{gray}{rgb}{0.5,0.5,0.5}
\definecolor{mauve}{rgb}{0.58,0,0.82}

\lstset
{
    frame=none,
    basicstyle={\small\ttfamily},
    commentstyle=\color{dkgreen},
    stringstyle=\color{mauve},
    numbers=none, %Nummerierung
    numberstyle=\tiny\color{gray},
    showstringspaces=false,
    breaklines=true,
    aboveskip=3mm,
    belowskip=3mm,
    columns=flexible,
    breakatwhitespace=true
}

\lstdefinelanguage{csharp}
{
    language=[Sharp]C,
    keywordstyle=\color{blue},
    tabsize=3,
    morekeywords={partial, var, value, get, set}
}

\lstdefinelanguage{json}
{
    stepnumber=1,
    numbersep=8pt,
    string=[s]{"}{"},
    comment=[l]{:\ "},
    morecomment=[l]{:"},
    literate=
        *{0}{{{\color{numb}0}}}{1}
         {1}{{{\color{numb}1}}}{1}
         {2}{{{\color{numb}2}}}{1}
         {3}{{{\color{numb}3}}}{1}
         {4}{{{\color{numb}4}}}{1}
         {5}{{{\color{numb}5}}}{1}
         {6}{{{\color{numb}6}}}{1}
         {7}{{{\color{numb}7}}}{1}
         {8}{{{\color{numb}8}}}{1}
         {9}{{{\color{numb}9}}}{1}
}

\hypersetup{
    colorlinks=true,
    linkcolor=black,
    filecolor=magenta,
    urlcolor=blue
}
% changes font

\bibliography{wipro-main-doc}
\graphicspath{ {images/} }

\newcommand{\tabitem}{~~\llap{\textbullet}~~}
\newcommand{\rot}{\rotatebox{90}}

% Set name of image label
\renewcommand{\figurename}{Abbildung}

\title{
    {Wirtschaftsprojekt} \\
    \vspace{10mm}
    { Herbstsemester 2022 } \\
    \vspace{10mm}
    % https://commons.wikimedia.org/wiki/File:HSLU_2022_logo.svg
    {\includegraphics[width=75mm]{img/hsluLogo2022.png}}
}

\author{Yannis Kr\"amer und Nicolas Wiedmer}
% TODO: Update date
\date{21.12.2022}

\begin{document}

\maketitle

\newpage

\noindent
\fontsize{12}{14}
\textbf{Wirtschaftsprojekt an der Hochschule Luzern -- Informatik} \\ \vspace*{0.6cm}

\fontsize{10.95}{12}
\noindent
\textbf{Titel:} STAIR Discord Bot \\ \vspace*{0.2cm}

\noindent
\textbf{Studentin/Student:} Yannis Kr\"amer \newline \newline
\textbf{Studentin/Student:} Nicolas Wiedmer \newline \newline
\textbf{Studiengang:} BSc Informatik oder Wirtschaftsinformatik  \newline \newline
\textbf{Jahr:} 2022 \newline \newline
\textbf{Betreuungsperson:} Markus Waldmann \newline \newline
\textbf{Expertin/Experte:} \newline \newline
\textbf{Auftraggeberin/Auftraggeber:} STAIR (Martin Steiger \& Estefania Otero)\newline \newline \newline
\textbf{Codierung / Klassifizierung der Arbeit:}\\
$\boxtimes$ \"Offentlich
$\square$ Vertraulich


%%% you can use \boxtimes for filling a cross inside the square
%%% e.g., $\boxtimes$ A: Einsicht 	(Normalfall)


\paragraph{\textbf{Eidesstattliche Erkl\"arung}}
Ich erkl\"are hiermit, dass ich/wir die vorliegende Arbeit selbst\"andig und ohne unerlaubte fremde Hilfe angefertigt haben, alle verwendeten Quellen, Literatur und andere Hilfsmittel angegeben haben, w\"ortlich oder inhaltlich entnommene Stellen als solche kenntlich gemacht haben, das Vertraulichkeitsinteresse des Auftraggebers wahren und die Urheberrechtsbestimmungen der Hochschule Luzern respektieren werden. \newline \newline
Ort / Datum, Unterschrift	\underline{\hspace*{8cm}} \newline \newline
Ort / Datum, Unterschrift	\underline{\hspace*{8cm}} \newline \newline \newline
\textbf{Ausschliesslich bei Abgabe in gedruckter Form: \\
Eingangsvisum durch das Sekretariat auszuf\"ullen}\newline \newline
Rotkreuz, den	\underline{\hspace*{4cm}}	\hspace*{1cm} Visum:	\underline{\hspace*{4cm}}

\normalfont
\newpage
\section*{I{\hspace*{1cm}}Abstract}
\todo{schreiben}

\newpage

\tableofcontents

\newpage

\section{Problem, Fragestellung, Vision}
Die bisherige Lösung ist instabil und führt regelmässig zu Problemen.
So zum Beispiel funktionieren gewisse Befehle von Nadeko nicht mehr.
Was dazu geführt hat ist nicht bekannt.
Der Stan-Bot hat keine Bugs, jedoch gibt es hin und wieder unklarheiten bei den Studenten,
weshalb diese nachfragen in den Hilf-Channels auf dem Discord Server.
In der neuen Version sollten die nächsten Schritte und Abläufe für die Studenten verdeutlicht werden.
\newline
Die beiden bisherigen Lösungen sollen vereinheitlicht werden, was langzeitig Zeit bei der Übergabe an die nächsten Vorstandsmitglieder sparen soll.
Auch hat die jetztige Lösung einige Einschränkungen.
Da die Anzahl Rollen auf Discord beschränkt ist, mussten bisher mehrere Fächer auf einmal an Studenten freigegeben werden.
Die Module sind dabei auch ungleichmässig verteilt.
So gibt zum Beispiel die Rolle Security 14 Module auf einmal frei, die Rolle PREN hingegen nur vier Module und die Rolle AD nur ein einziges Modul.
Hierbei sind vor allem die grössten Zusammenfassungen ein Problem, da dann Studierende mit zu vielen Nachrichten belästigt werden.
Diese negative Nutzererfahrung soll zukünftig verhindert werden.
\newline
Zukünftig sollten Statistiken für den Vorstand von STAIR verfügbar sein um mehr über die Nutzung des Discord Servers zu erfahren
und datenbasierte Entscheidungen treffen, sowie Werbung bei den Studentengruppen einsetzen zu können.
\newline
Neu soll eine Fehlererkennung eingeführt werden.
Bisher wurde kein Admin informiert bei Problemen.
Dies führte zu länger anhaltenden Fehlern als nötig.
Mit einer automatischen Erkennung, zum Beispiel von der nicht-erreichbarkeit des Mail-Servers, Datenbank oder Discord-Servers könnte ein Admin umgehend informiert werden um Massnahmen zu treffen.
Hierzu würde der jeweils andere Kommunikationskanal genutzt werden, also bei einem Fehler mit Discord wird per E-Mail informiert und umgekehrt.
Es kann auch auftreten, dass eine Verbindung zu der Datenbank möglich ist, jedoch können keine Daten gespeichert werden.
\todo{Decide if this will be done or not}
Auch wenn jemand versucht für ein Modul anzumelden, welches nicht existiert, sollte dies erfasst werden um mögliche Fehler zu erkennen.
\todo{Decide if this will be done or not}
Ein weiterer Fall wäre wenn jemand versucht mit einer ungültigen E-Mail sich anzumelden.
Hierbei wird unterschieden zwischen nicht erfasster Studenten Mail und einer nicht-studenten Mail.
\todo{Format Studenten Mail irgendwo einfügen}
Mit Hilfe einer Logging Library sollen Fehler nachvollziehbar bleiben.
Hierzu kann eine bestehende Lösung genutzt werden, da keine speziellen Features gebraucht werden
und ein lokales Speichern der Log Messages ist ausreichend.

\newpage
\section{Stand der Technik}

\subsection{Sprach- und Text-Messenger}

\subsection{Discord}
Der Online Dienst Discord ist ein Instant-Messaging und Chat Tool mit Sprach- und Videokonferenz Funktion.
Er kann online auf einer Webseite, oder mit Hilfe eines Clients lokal aufgerufen werden.
Discord unterst\"utzt alle g\"angigen Betriebssysteme und kann auch auf mobilen Endger\"aten verwendet werden.

\subsubsection{Geschichte}
Urspr\"unglich wurde Discord f\"ur Computerspiele geschaffen, um die Kommunikation zwischen den Spielkameraden zu vereinfachen und zu verbessern.
Das Ziel war es, neben der komplexen Spielmechanik einen Chat-Messanger zu bauen, der benutzerfreundlich und effizient erweist.
2012 wurde das Unternehmen Discord Inc (damals noch Hammer \& Chisel)
als Startup gegr\"undet.\autocite{noauthor_discord_2021}
Im Jahre 2014 konnte die Unternehmung Hammer \& Chisel f\"ur die Weiterentwicklung Ihrer
Applikation auf zus\"atzliche Finanzierungsmittel, von anderen Unternehmen zur Verfügung gestellt, zurückgreifen.

2015 wurde Discord unter der Domain "discordapp.com" ver\"offentlicht.
Der Text und Sprach Messanger erfreute sich schnell grosser Beliebtheit
und geriet so in das Blickfeld grösserer Investoren wie z.B. Warner Media, die den Dienst 2016 mit rund
20 Millionen US-Dollar unterstützte. \autocite{noauthor_warner_2022} .
2018 k\"undete auch Microsoft Discord Unterst\"utzung f\"ur ihre XBox Live Nutzer an und unterst\"utzte Discord mit einer Finanzierung
von 150 Millionen US-Dollar. Bewertet wurde Discord Inc nun auf etwa 2 Milliarden US-Dollar.

Aufgrund der Covid-19 Pandemie und den steigenden Benutzerzahlen stellte sich Discord auch f\"ur andere Zielgruppen ein.
So wurde der Fokus von Videospielen weggelenkt auf einen universellen Kommunikations-und Chat Client, um es f\"ur andere Branchen,
wie das Schulwesen oder innerhalb Unternehmen, ansprechender zu machen.
Damit \"anderte das Unternehmen ihr Motto von \textit{Chat for Gamers} zu
\textit{Chat for Communities and Friends} und f\"uhrte Servervorlagen ein.

Heute hat Discord mehr als 140 Millionen monatlich aktive User, verwaltet etwa 13.5 Millionen aktive Server und
wird auf 15 Milliarden US-Dollar gewertet (Stand 2021). \autocite{david_curry_discord_2022}

\subsubsection{Discord Einschr\"ankungen}\label{discord_einschraenkungen}

% https://netcord.site/new-discord-channel-creation-ratelimit/
% https://support.discord.com/hc/en-us/community/posts/360056762431-Increase-channel-limit
% https://support.discord.com/hc/en-us/community/posts/360032363631-Increase-the-Max-Number-of-Roles-per-Server

\subsubsection{Alternativen zu Discord}
\todo{Yannis}

\subsubsection*{Slack}

Slack ist eine sehr ähnliche Alternative zu Discord, richtet sich jedoch mehr an Unternehmen als an Communities.
Dies ist auch gut auf der Homepage des jeweiligen Unternehmens zu erkennen.
So erwähnt Discord zum Beispiel: "\textit{STELL DIR EINEN ORT VOR, …
… an dem du Teil eines Schulklubs, einer Gaming-Gruppe oder einer weltweiten Kunst-Community sein kannst. Ein Ort, an dem du einfach Zeit mit Freunden verbringen kannst. Ein Ort, an dem es leicht ist, sich jederzeit zu treffen und zu unterhalten.}" \todo{source: www.discord.com}
wo hingegen Slack folgendes schreibt: "\textit{Great teamwork starts with a digital HQ
With all your people, tools and communication in one place, you can work faster and more flexibly than ever before.}" \todo{Source: slack.com}
Auch der Fokus der Anwendung liegt stärker auf dem Textchat als auf dem Sprachchat, wo sich Discord widerum mehr fokusiert.
Da STAIR und der Discord Server auch für die Freizeit gedacht sind, macht es mehr Sinn auf ein Freizeit Tool zu setzen.
\todo{add source}

\subsubsection*{Skype}
Skype ist eine ältere und sehr verbreitete Sprachchat Lösungen mit 100 Millionen aktiven Nutzern jeden Monat.
\todo{source: https://techcrunch.com/2020/03/30/microsoft-teams-is-coming-to-consumers-but-skype-is-here-to-stay/}
Skype ist jedoch auch sehr eingeschränkt, was die Features angeht.
So sind keine echten Communities möglich und Sprachchat ist nur per Anruf möglich.
So können neue Nutzer also nicht einfach zu einem Sprachchat hinzukommen.
Richtige Communities sind ebenfalls nicht vorhanden.
\todo{source: https://www.gamingscan.com/discord-vs-skype-for-gaming/}
Diese fehlenden Features machen das Programm ungeeignet als Alternative.
\todo{Source:}

\subsubsection*{TeamSpeak}

TeamSpeak macht es möglich einfach und ohne viel Rechenressourcen einen Server zum Austauschen auf dem eigenen Server aufzusetzen.
Auch wenn Textchats möglich sind, sind diese stärker eingeschränkt und klar nicht der Fokus der Anwendung.
\todo{check following statement}
Da Textchat der stärkere Fokus ist für den Austausch den STAIR anbieten will,
ist dies wohl keine geeignete Lösung.

\subsubsection*{Matrix}

Matrix wurde 2019 veröffentlicht und ist eine Open Source Alternative zu Discord.
Hierbei kommt Matrix dem Discord mit der Bedienung sehr nahe.
Ein starker Fokus liegt auf der Privatsphäre der Nutzer.
So ist man unabhängig von den Entwicklern und kann den Matrix Server auf dem eigenen Server aufsetzen.
Die Software bringt jedoch auch seine Nachteile mit sich.
So ist die Verbreitung noch relativ klein und hat wenig Nutzer.
Mit 17 Millionen Nutzer \todo{source: https://matrix.org/faq/} hinkt Matrix noch stark an den 300 Millionen Nutzern von Discord hinten nach. \todo{source: https://www.businessofapps.com/data/discord-statistics/}
So müssten viele Studierende zuerst einen Account erstellen und sich einen Matrix aussschliesslich für den STAIR Server herunterladen. Dies wäre eine zusätzliche Hürde.
Auch wäre das betreiben des Matrix Servers umständlicher, da die Updates neu manuell installiert werden müssen anstatt dies von Discord machen zu lassen.
Auch wäre unklar, ob das Enterprise Lab eine solche Applikation zulassen würde, da dies einiges an Bandbreite kosten würde, was Discord zurzeit übernimmt für STAIR.

Der Vorteil an Matrix ist, dass jeder seinen eigenen Client wählen oder sogar entwickeln kann. \todo{source: https://matrix.org/clients}
So sollte jeder Nutzer etwas passendes finden.
Auch ist die Privatsphäre der Studierenden geschützt, da keine Informationen nach aussen geraten wenn der Server selbst gehostet ist.

\subsubsection*{Warum Discord?}
Wie man gesehen hat, gibt es noch andere Sprach- und Text Messanger auf dem Markt.
Für dieses Projekt ist Discord verbindlich festgelegt.
STAIR kommuniziert schon lange über die Plattform und es hat sich eine grosse Community mit knapp 700 Studierenden und Exstudierenden aufgebaut.
Discord erlaubt grosse Freiheiten in der Erstellung von Servern und Channels, deren Bearbeitung und der Freigabe unter den Mitgliedern.
Da es ein kostenloser Dienst ist und diese Einstellungsmöglichkeiten hat, ist es sehr verbreitet bei jungen Leuten aller Ethnien und Gruppierungen.

\newpage
\section{Ideen und Konzepte}

Bei der Konzeptbesprechung wurde ein Entity-Relationship Diagramm ausgearbeitet. Das Diagramm dient als Grundlage für die Design-
und Architektur Entscheide der neuen Applikation.
\todo{Bild ER-Diagramm}

Beschrieb ER-Diagramm


Überblick der Software mit den verschiedenen Scripts

\subsection{Technologien}
\todo{Yannis}

\subsubsection{C\# und .NET Framework}

Es wurde entschieden weiterhin mit C# zu arbeiten.
Dies wurde auch für die bereits existierende Version verwendet.
Discord Libraries sind jedoch auch in anderen programmiersprachen verfügbar.
Discord selbst führt eine Liste zu den verfügbaren Libraries und deren unterstützten Programmiersprachen, welche von der Community entwickelt wurden.\autocite{noauthor_discord_2022-1}
Bei Besprechungen mit den Stakeholdern wurde kein Grund gesehen auf etwas anderes umzusteigen, vor allem da durch das existierende Projekte eine gewisse Machbarkeit für einen Teil der Features bereits bewiesen wurde.
Durch die neuen .NET Versionen wird es nun auch möglich den C# Code auf andere Platformen als nur Windows zu exportieren.
Hierfür wurde früher Project Mono gebraucht oder .NET Core.
Die beiden Projekte wurden mit dem .NET Framework zusammengeführt zu der .NET 5 Version.\autocite{schwichtenberg_net_2019}
Im November 2022 ist die neue .NET 7 Version herausgekommen.
Da .NET 6 am aktuellsten war zum Startzeitpunkt des WIPRO Projektes, die .NET 6 Version eine LTS Version ist und damit länger als .NET 7 unterstützt wird und in der Praxis stärker getestet ist, wird die WIPRO mit .NET 6 zuende geführt.\autocite{noauthor_net_2022}
Zur Entwicklungszeit der alten Stan Discord Bot Version wurde noch auf .NET Framework gesetzt, weshalb der Bot auch noch nicht auf Linux lauffähig ist.
Dies war jedoch der Wunsch von STAIR für zukünftige Projekte.

\todo{siehe Kapitel "Geplante Änderungen}

\subsubsection{MySQL}

Als Datenbank wurde MySQL ausgewählt.
MSSQL hat die beste Unterstützung in C\# jedoch ist diese Kostenpflichtig.
Es gibt eine kostenfreie Version, jedoch ist diese auf 10 GB limitiert.
Dies macht die Technologie unattraktiv im Vergleich mit anderen Technologien ohne diese Einschränkungen und Kosten.\autocite{noauthor_sql_nodate}

% https://www.nuget.org/packages/linq2db.MySql/
% http://www.primaryobjects.com/2009/01/24/using-mysql-and-linq-to-sql-in-c-asp-net/

\subsection{Teststrategie}\label{Teststrategie}
In diesem Projekt wird die abzugebende Software iterativ entwickelt.
Ein genauerer Beschrieb dazu ist im Kapitel \nameref{Vorgehensmodell} zu finden.
Diese Vorgehensweise erlaubt es flexibel auf neue Anforderungen zu reagieren und
während jeder Phase die Funktionalität sicher zu stellen.
\newline
Die einzelnen Komponenten innerhalb der Software werden mithilfe von Unit-Tests getestet.
Diese Tests werden nach dem AAA-Pattern "Arrange, Act, Assert" erstellt und ausgeführt. \autocite{noauthor_arrangeactassert_nodate}
Vor einem Meilenstein oder der Übergabe der Software wird die Software auch einigen manuellen Systemtests unterzogen.
\newline
Das Drehbuch für die manuellen Systemtests findet man im Kapitel \nameref{Testdrehbuch}.

\newpage
\section{Methoden}

\subsection{Projektführung}
\todo{Nicolas}

\subsubsection{Projektart}
Es gibt verschiedene Projektarten, welche je nach Vorhaben zur Anwendung kommen. Eine Projektart hilft, die erwarteten Ergebnisse zu spezifizieren
und geht auch mit der richtigen Wahl eines Vorgehensmodells daher.
\newline
Ein Projekt kann nach verschiedenen Modellen typisiert werden. Verbreitete Ansätze dabei sind:
\begin{itemize}
    \item Portfoliobezogene Projektklassifikation
    \item Externe und interne Projekte
    \item Projektarten nach Trägern
    \item Unterteilung nach Komplexität von Projektinhalt und Projektumwelt
    \item Diamond Approach
    \item oder die Erstellung eines Projektprofils\autocite{claus_husselmann_zielgerichtete_nodate} %PMRE Projekttypisierung
\end{itemize}
Auf den genaueren Beschrieb der einzelnen Modelle, wird verzichtet, da für dieses Projekt eine Vorgabe der verschiedenen Projektarten gemacht wurde.
Für zukünftige Projekte könnte man aber auf diese Projekttypisierungsmodelle zurückgreifen um sein Vorhaben genau einzustufen.
\newline

Für dieses Wirtschaftsprojekt standen folgende Projektarten zur Auswahl:
\begin{itemize}
    \item Einsatz von Standardsoftware und Services
    \item Software- und Produktentwicklung
    \item Innovationsprojekte (Projekte mit Erkenntnisgewinn, Forschungsprojekte)
    \item IT-Infrastrukturentwicklung
    \item Strukturierte Analyse und Konzeption von Systemen und Abläufen \autocite{oliver_gilbert_wipro_2022} %Wegleitung
\end{itemize}

Unsere Aufgabestellung verlangt, dass wir in einer ersten Phase, eine Analyse der bestehenden Infrastruktur und des zur Zeit laufenden Bots durchführen.
Die Analyse beruht auf der Frage, ob die zurzeit laufende Applikation, den neuen Features angepasst werden kann oder ersetzt werden muss.
In einer zweiten Phase wird evtl. ein neuer Bot in einer neuen Infrastruktur implementiert oder nur die neuen Features hinzugefügt.
In einer dritten Phase wird eine Anleitung für den zukünftigen Unterhalt des Discord-Servers und des Bots erstellt.

Aufgrund dieser Aufgabenstellung wurde die Projektart "Software- und Produktentwicklung" gewählt. Da wir eine Analyse von bestehender Software machen und
entweder bestehende Software weiterentwickeln oder eine neue erstellen.
\newpage
\subsubsection{Vorgehensmodell}\label{Vorgehensmodell}
"\textit{Ein Vorgehensmodell ist eine mehr oder weniger genaue Anleitung, in welchen Schritten und durch welche Tätigkeiten das Projektziel
erreicht werden kann.}"\autocite{sarre_lufthansa-reservierung_2009}
Es beschreibt die verschiedenen Projektphasen, Meilensteine, Rollen, Aufgaben und die Arbeitsergebnisse (Artefakte) unter einheitlichen Begriffen.
Des Weiteren dient es mit verschiedenen Methoden, Techniken, Standards und kann die Übersichtlichkeit und Planbarkeit in einem Projekt stark erhöhen.
Bei Nutzung eines Vorgehensmodells ist die Wahrscheinlichkeit ein Projekt innerhalb der festgelegten Zeit, mit dem verfügbarem Budget und in einer
angemessenen Qualität fertigzustellen, insgesamt grösser. \autocite{jenny_projektmanagement_2016} %PMB Projektmanagement folie23
\newline

Man unterscheidet grundlegend zwischen klassischem und agilem Projektmanagement.
Beim klassischen Vorgehensmodellen arbeitet man meist sequentiell.
Also eine Phase folgt der anderen und baut darauf auf.
Rückkopplungen sind meistens nicht möglich und verzögern den Endtermin des Projektes.\\
Bespiele für klassische Vorgehensmodelle sind
\begin{itemize}
    \item das Wasserfallmodell, in dem die Phasen die verschiedenen Aktivitäten darstellen.
    \item das V-Modell, welches man gut für die Qualitätssicherung brauchen kann.
    \item oder Hermes, welches vor allem vom Schweizer Bund verwendet wird.
    \item etc.
\end{itemize}

Im agilen Projektmanagement arbeitet man iterativ und kann in den einzelnen Phasen immer wieder neu planen.
Dies erlaubt eine schnellere Rücksprache mit den Stakeholdern und die Einbringung neuer Ideen. \\
Beispiele für iterative Vorgehensmodelle sind
\begin{itemize}
    \item Scrum, welches den Grundstein des agilen Projektmanagement gebildet hat.
    \item SAFe steht für (Scaled Agile Framework) und erlaubt es Scrum in grossen Organisationen einzusetzen.
    \item etc. \autocite{noauthor_liste_2022}
\end{itemize}

Je nach Teamgrösse und Vorhaben eignet sich eher ein klassisches oder agiles Vorgehen.
Klassische Vorgehen werden meist bei sehr grossen Projekten eingesetzt, wie Bau- und Infrastruktur Projekten oder
innerhalb Wertschöpfungsketten. Also dort, wo es eine lange Planungsphase braucht und es keinen Sinn macht iterativ vorzugehen.
Das agile Vorgehen wird meist eher in kleineren Teams verwendet und ist vor allem in der Software-Entwicklung oder
im Marketing bekannt.
\newline

Für dieses Projekt wird SoDa (Software Development Agile) verwendet und ist ein hybrides Vorgehen,
das bedeutet ein Mix aus klassischem und agilem Vorgehen.
Es besteht auch aus verschiedenen Phasen die untereinander abgeschlossen sind,
hat aber auch einen iterativen Teil in der Konzeptions- und Realisierungsphase, bei dem man nach Scrum vorgeht.
\newpage

\begin{figure}[h]
    \centering
    \includegraphics[width=1.0\textwidth]{img/SoDa.png}
    \caption{Software Development Agile}
    \label{fig:SoDa}
\end{figure}


Wie der Name schon sagt, ist es für die Software Entwicklung konzipiert und erlaubt es, die gängigen Artefakte,
welche man bei der Software Entwicklung erstellen muss, in die Projektphasen miteinzubinden.
So wird in der Initialisierungsphase der Projektauftrag, Business-Case und der Anfroderungskatalog definiert und
in der Einführungsphase können Anleitungen oder Einführungen für den operativen Einsatz erstellt werden.
Die Konzeptions- und Realisierungphase erlaubt es, die Vorteile von der agilen Vorgehensweise nach Scrum,
in der Entwicklung zu gebrauchen. Durch das Vorgehen mithilfe von Sprints können bei jedem Zyklus neue
Anforderungen erfasst und das weitere Vorgehen geplant werden. Die Sprints können beliebig lang gesetzt werden,
im Geschäftsumfeld üblich sind aber 2-Wochen.


Die verschiedenen Phasen unserer Aufgabenstellung lassen sich gut auf die verschiedenen Phasen von SoDa abbilden.
In der Initialisierungsphase wird die Analyse der jetzigen Software erstellt und das Projekt geplant.
In der Konzeptiopns- und Realisierungsphase wird in Sprints eingeteilt, die neue Software erstellt.
Und in der Einführungsphase kann die Bedienungsanleitung für den weiteren Gebrauch geschrieben werden.

\subsubsection{Rahmenplan}
Der Rahmenplan gibt ein Überblick über die ganzen Phasen des Projekts.
Die geplanten Zwischenergebnisse werden an fixen Punkten im Projekt als Meilensteine deklariert und mit konkretem Datum versehen.

\begin{figure}[h]
    \centering
    \hspace*{-2cm}
    \includegraphics[width=1.3\textwidth]{img/Rahmenplan.jpg}
    \caption{Rahmenplan}
    \label{fig:Rahmenplan}
\end{figure}
Es wurde entschieden eine Intialisierungsphase von 3 Wochen zu planen, da die Analyse der bestehenden Software in dieser Phase stattfindet.
Am Schluss bleibt noch eine Woche für die Einführungsphase und das Erstellen der Anleitung.
Für die Realisierungsphase bleiben 9 Wochen Zeit, die in Sprints eingeteilt werden können.
Es verbleibt ein einwöchiger Sprint und 4 reguläre zweiwöchige Sprints.

\subsubsection*{Meilensteine}
Meilensteine erlauben es den Projektfortschritt festzustellen,
in dem zuvor definierte Projektergebnisse (Artefakte) an einem gewissen Datum vorliegen.\\
Artefakte sind konkrete Dokumente oder Software die vorliegen muss. Zum Beispiel:
\begin{itemize}
    \item Testprotokolle
    \item Prototypen
    \item Software Releases
    \item Sprintplannungen
    \item etc.
\end{itemize}

In SoDa ist es normal, bei jedem Phasenwechsel ein Meilenstein zu definieren und in der Mitte der
Realisierungsphase noch einmal. \autocite{jenny_projektmanagement_2016} % PMB Projektplanung p23-25
Nach diesem Vorgehen erhält man 5 Meilensteine für dieses Projekt.

\begin{table}[h]
    \centering
    \begin{tabular}{|l|l|l|}
        \hline
        \rowcolor[gray]{.9} MS & Datum & Artefakte \\
        \hline
        1 & 19.09.2022 & \tabitem Aufgabenstellung \\
        \hline
        2 & 09.10.2022 & \tabitem Architekturdokument \\
         & & \tabitem Analyse \\
         & & \tabitem Testkonzept \\
         & & \tabitem Sprintplannung 1 \\
        \hline
        3 & 13.11.2022 & \tabitem Software-Release 0.5 \\
         & & \tabitem Sprintplannung 4 \\
        \hline
        4 & 12.12.2022 & \tabitem Software-Release 1.0 \\
         & & \tabitem Dokumentation Realisierung \\
        \hline
        5 & 18.12.2022 & \tabitem Bedienungsanleitung \\
        \hline
    \end{tabular}
    \caption{Meilensteine}
    \label{tab: Meilensteine}
\end{table}
\clearpage
\subsection{Risikomanagement}
Das Ziel jedes Projektes ist, eine möglichst hohe Wertschöpfung zu generieren.
Doch Wertschöpfung und Risiko stehen komplementär zueinander.
Das bedeutet, je höher die Wertschöpfung, desto höher das Risiko.
Um die Wertschöpfung möglichst hoch zu halten und die Risiken zu minimieren wird Risikomanagement betrieben.
Im Risikomanagement unterscheidet man zwischen \textbf{Produktrisiken} und \textbf{Projektrisiken}.
\newline
Produktrisiken werden direkt als Arbeitspakete im Projekt verbaut.
Dabei wird geschaut, welche Gefahren für Mensch und Umwelt während der ganzen Umsetzung und Betreibung auftreten können.
\newline
Die Projektrisiken werden im Projektmanagement behandelt.
Dabei wird geschaut, welche Probleme mich daran hindern könnten ein Projekt erfolgreich abzuschliessen.
Dies könnten sein:
\begin{itemize}
    \item technische Risiken
    \item Implementierungsrisiken
    \item wirtschaftliche, industrielle und Geschäftsrisiken \autocite{} % PMB Risikomanagement p8-10
\end{itemize}
\noindent
Das Ziel des Risikomanagements ist es, Risiken frühzeitig zu erkennen und entsprechende Massnahmen zu ergreifen.
Der Prozess dabei ist folgender:
\begin{enumerate}
    \item Identifizieren
    \item Analysieren
    \item Priorisieren
    \item Massnahmen erarbeiten
    \item Überwachen \autocite{} % PMB Risikomanagement p16
\end{enumerate}
\noindent
In diesem Kapitel werden verschiedene Grundrisiken identifiziert, die während dem Projekt auftreten könnten.
Da dieses Projekt während der Umsetzungsphase iterativ läuft, kann bei den \nameref{Sprintreviews}, ein Risiko-Update durchgeführt werden.

\begin{longtable}[h]{|p{1em}|p{8em}|p{10em}|p{7em}|p{5em}|p{1em}|}
    \hline
    \rot{ID} & \rot{Risiko} & \rot{Beschreibung} & \rot{\shortstack[l]{Eintritts-\\wahrscheinlichkeit}} & \rot{Schadensausmass} & \rot{Risikoskala} \\*
    \hline
    R1 & Verbindung zwischen Bot und Discord bricht ab. &
    Die Verbindung zwischen dem Server (Discord) und dem Client (Bot) ist nicht mehr gewährleistet.
    Dies tritt ein, wenn der Server auf dem der Bot läuft, keine Verbindung zum Internet mehr hat.
    Dadurch können keine Daten und Befehle ausgetauscht werden. &
    Möglich & Hoch & 6 \\
    \hline
    R2 & Das Einlesen der Listen zerstört die Datenbank & Die Studenten- oder Modulliste, welche in das System eingelesen wird,
    enthält irgendwelche Escape-Characters.
    Dadurch könnte es beim Eintragen in die Datenbank zu Problemen kommen und diese im schlimmsten Fall zerstören. &
    Unwahrscheinlich & Kritisch & 4 \\
    \hline
\end{longtable}

% Library nicht alle Funktionen die benötigt werden
% Fehlermeldungen an Admin weiterleiten
% Absturz des Servers Enterprise Lab

\begin{tabular}[h]{ll}
    \textbf{Eintrittswahrscheinlichkeit} & \textbf{Schadensausmass} \\
    \tabitem Unwahrscheinlich (1) & \tabitem Gering (1) \\
    \tabitem Möglich (2) & \tabitem Mittel (2) \\
    \tabitem Wahrscheinlich (3) & \tabitem Hoch (3) \\
    \tabitem Sehr wahrscheinlich (4) & \tabitem Kritisch (4) \\
\end{tabular}

\clearpage
\subsubsection{Massnahmen}
\noindent
Nun werden die Risiken analysiert und entsprechende Massnahmen erarbeitet.
Dabei wird das neue Risiko mit angepasster Eintrittswahrscheinlichkeit und Schadensausmass festgehalten.
\begin{longtable}[h]{|p{1em}|p{8em}|p{10em}|p{7em}|p{5em}|p{1em}|}
    \hline
    \rot{ID} & \rot{Risiko} & \rot{Massnahmen} &
    \rot{\shortstack[l]{Eintritts-\\wahrscheinlichkeit\\(neu)}} &
    \rot{\shortstack[l]{Schadensausmass\\(neu)}} &
    \rot{\shortstack[l]{Risikoskala\\(neu)}} \\
    \hline
    R1 & Verbindung zwischen Bot und Discord bricht ab. &
    Die Verbindung zwischen dem Server (Discord) und dem Client (Bot) ist nicht mehr gewährleistet.
    Dies tritt ein, wenn der Server auf dem der Bot läuft, keine Verbindung zum Internet mehr hat.
    Dadurch können keine Daten und Befehle ausgetauscht werden. &
    Möglich & Hoch & 6 \\
    \hline
    R2 & Das Einlesen der Listen zerstört die Datenbank & Die Studenten- oder Modulliste, welche in das System eingelesen wird,
    enthält irgendwelche Escape-Characters.
    Dadurch könnte es beim Eintragen in die Datenbank zu Problemen kommen und diese im schlimmsten Fall zerstören. &
    Unwahrscheinlich & Kritisch & 4 \\
    \hline
\end{longtable}

\subsubsection{Risikomatrix}
Die Risiken werden in eine sogenannte Risikomatrix eingetragen, wobei die Achsen die Eintrittswahrscheinlichkeit und das Schadensausmass repräsentieren.
In dieser Matrix kann die Risikobereitschaft der Projektverantwortlichen eingetragen werden.
Die Risikobereitschaft wird mit einer Linie durch die Matrix dargestellt.
Wenn ein Risiko über der Linie liegt, müssen zwingend bessere Massnahmen ergriffen werden.
\newline
Das Ziel ist es mit den getroffenen Massnahmen alle Risiken unter diese Risikobereitschaftslinie zu bringen.

\begin{table}[h]
    \centering
    \begin{tabular}{|l|p{2cm}|p{2cm}|p{2cm}|p{2cm}|}
        \hline
        \shortstack[c]{Schadensausmass / \\ Eintrittswahrscheinlichkeit} & Gering & Mittel & Hoch & Kritisch \\[10pt]
        \hline
        Sehr wahrscheinlich & \cellcolor{yellow!50} & \cellcolor{orange!50} & \cellcolor{red!50} & \cellcolor{red!50} \\[10pt]
        \hline
        Wahrscheinlich & \cellcolor{yellow!50} & \cellcolor{yellow!50} & \cellcolor{orange!50} & \cellcolor{red!50} \\[10pt]
        \hline
        Möglich & \cellcolor{green!50} & \cellcolor{yellow!50} & \cellcolor{yellow!50} & \cellcolor{orange!50} \\[10pt]
        \hline
        Unwahrscheinlich & \cellcolor{green!50} & \cellcolor{green!50} & \cellcolor{yellow!50} & \cellcolor{yellow!50} \\[10pt]
        \hline
    \end{tabular}
\end{table}

\clearpage
\section{Realisierung}

\subsection{Analyse der bestehenden Infrastruktur}
Als erster Schritt wird eine Analyse des bestehenden Bots und seiner Funktion im Discord durchgef\"uhrt.

\subsubsection{Aufbau Discord}
Der Discord Server "STAIR" wird von STAIR verwaltet und ist ein online Treffpunkt f\"ur alle Studenten des Departements Informatik.
Beim erstmaligen Eintretten in den Server hat man noch keine Berechtigung und es wird deshalb noch nichts, ausser dem Help Channel
angezeigt. Man bekommt vom Bot Stan, beim Eintretten auf den Server, eine Nachricht mit einer Anleitung. Darin ist beschrieben, wie
man sich authentifizieren kann. Bei der Authentifizierung wird vom Bot geschaut ob es sich um eine Studenten E-Mail Adresse handelt.
Wenn dem so ist, schickt der Bot einen 6-stelligen Random Code, den der Benutzer im Discord dem Bot zur\"uckschreiben muss.

Hat dieser Prozess funktioniert, ist man "eingeloggt" und bekommt vom Bot die Rolle Student zugeteilt. Damit hat man Zugriff auf
die verschiedenen Grundchannels, wie Gaming, Administration, General, Studying, etc.
In diesen Channels kann man sich mit seinen Mitstudierenden Sprachlich oder per Text \"uber alle Themen austauschen.
\newline

Der STAIR Discord-Server bietet auch Informationen und Unterst\"utzung f\"ur alle Module an. Dabei hat jedes Modul einen eigenen Channel.
Dort k\"onnen sich Studierende gezielt \"uber ein Modul austauschen, Fragen stellen oder Informationen mitteilen.
Am Anfang wird gar nichts angezeigt. Man muss sich spezifisch f\"ur die Module registrieren um diese angezeigt zu bekommen. Dies verhindert

\begin{enumerate}
    \item dass man unn\"otig Benachrichtigungen von Channels bekommt, die einen nicht interessieren.
    \item dass sein Discord nicht \"uberf\"ullt wird mit 400 Modul-Channels.
\end{enumerate}

Die Commands \textit{show <module>} und \textit{hide <module>} erlauben es, den Channel hinzuzuf\"ugen oder zu entfernen.
\newline

Neu gibt es auch das Konzept von verschiedenen H\"ausern bei STAIR. Jeder Student wird dabei einem Haus zugeteilt. Die Zuteilung l\"auft
dabei \"uber das Sekretariat der Hochschule. Es wird geschaut, dass in jedem Studiengang in jedem Startsemester, die Studenten gleichm\"assig
auf die H\"auser verteilt werden.
Es gibt die H\"auser Blue, Purple, Red, Orange, Yellow und Grey. \"uber das Semester hinweg organisiert STAIR verschiedene Events, bei denen
man Punkte f\"ur sein Haus sammeln kann. Am Ende jedes Jahres, also Ende Fr\"uhlingssemster, wird das Haus, welches am meisten Punkte
gesammelt hat, als Gewinner gelobt.

Der Discord-Server bietet die Platform um sein Haus zu mobilisieren oder die neuesten Ergebnisse den Studenten zu verk\"unden.
Pro Haus gibt einen Channel. Momentan werden die Studenten noch manuell zu ihren zugeh\"origen Channels hinzugef\"ugt. \\
Im folgenden Bild sieht man eine Übersicht über alle Discord Channels und Kategroien, wie sie momentan bestehen.
\newpage
\begin{figure}[ht]
    \centering
    \includegraphics[width=1.0\textwidth,height=10cm]{img/Stair_Discord_Channels.jpg}
    \caption{Stair Discord Channels Alt}
    \label{fig:stair_old_discord_channels}
\end{figure}

\subsubsection*{Ablauf Authentifizierung \& Modulanmeldung}
In den folgenden zwei Seiten wird der Authentifizierungsprozess und der Modulanmeldungsprozess als Ablaufdiagramm beschrieben.
Dieser Ablauf wird zum jetzigen Zeitpunkt durchgeführt.
\todo{Noch mehr schreiben}

\begin{landscape}
    \begin{figure}[ht]
        \centering
        \hspace*{-4.1cm}
        \includegraphics[width=1.9\textwidth]{img/Authentifizierungsprozess_Bot_alt.png}
        \caption{Authentifizierungsprozess Bot alt}
        \label{fig:Authentifizierungsprozess_Bot_alt}
    \end{figure}
    \clearpage
    \begin{figure}[ht]
        \centering
        \hspace*{-4.1cm}
        \includegraphics[width=1.9\textwidth]{img/Modulanmeldungsprozess_Bot_alt.png}
        \caption{Modulanmeldungsprozess Bot alt}
        \label{fig:Modulanmeldungsprozess_Bot_alt}
    \end{figure}
\end{landscape}

\subsubsection{Funktionsweise Bot}
Der Bot ist mit der Programmiersprache C\# und dem .NET Framework geschrieben.

\subsubsection*{Konfigurationsmanagement alter Bot}
In der folgenden Tabelle ist eine \"Ubersicht der benutzen IDE, ihrer Spezifizierung und den eingesetzten Libraries.
\todo{Tabelle mit Spezifikationen vom System. (Versionen, Libraries)}
\begin{table}[h]
    \centering
    \begin{tabular}{|l|p{20em}|l|}
        \hline
        \rowcolor[gray]{.9} Name & Beschreibung & Version \\
        \hline
        .NET Core & Core Library für C\# Projekte.
        Enthält alle grundlegend Klassen und Libraries, welche man zum Arbeiten braucht. & 4.0.0 \\
        \hline
    \end{tabular}
    \caption{Konfigurationsmanagement}
    \label{tab: Konfigurationsmanagement}
\end{table}


Das Programm, welches den Bot zur Verfügung stellt, ist nicht gross. Es beinhaltet im Groben einen Client Socket der
Events on Discord abfängt und einer E-Mail Klasse, die E-Mails an den Benutzer versenden kann.
Das Programm ist in zwei Packages aufgeteilt.

\begin{itemize}
    \item StanBot.Core
    \item StanBot.Service
\end{itemize}

\begin{figure}[h]
    \centering
    \hspace*{-1.5cm}
    \includegraphics[width=1.2\textwidth]{img/Klassendiagramm_Bot_alt.png}
    \caption{Klassendiagramm Bot alt}
    \label{fig:Klassendiagramm_Bot_alt}
\end{figure}

\subsubsection{Ausführung Bot}

Der Bot läuft auf einem Windows Server \#\#\#\# auf der Enterprise Lab Umgebung der HSLU. Die Konfiguration entspricht der eines
Windows Services und läuft dementsprechend durchgehend.
\newline
\todo{Bild Service Konfiguration}
\newline


\subsubsection{Geplante Änderungen}
\todo{Yannis}

\subsubsection*{Wechsel zu Linux Server}

Es wurde zusammen mit dem STAIR Team entschieden,
auf Linux zu wechseln um alle Server einheitlich auf Linux zu haben.
\todo{Wordpress erklären}
Bei STAIR wird noch eine Wordpress-Seite und eine JavaScript Webseite betrieben.
Da die Projekte jedoch sich von einander stark unterscheiden wurde entschieden mit verschiedenen Servern zu arbeiten und nicht alle auf dem gleichen Server zu hosten.

\subsubsection*{Entscheid den Bot neu zu schreiben}

\todo{replace .NET Standard with .NET Framework version}
\todo{Add sources}
Bei der Analyse des bisherigen Bot wurde bemerkt, dass die veraltete Technologie .NET Standard 2.0 und .NET Core 3.1 mit veralteten Libraries genützt wurden.
Diese werden nicht mehr unterstützt und sollten ersetzt werden.
Hierbei wird .NET 6 mit C\# verwendet.
So kann man Teile des Codes wiederverwenden.
Die neueren .NET Versionen sind Cross-Platform fähig und können damit nun auch auf Linux verwendet werden.\autocite{de_george_installieren_nodate}
Die Sprache wird nicht an der Hochschule breit verwendet, es gibt jedoch ein Modul zu C\#\autocite{} \todo{how to add pdf??? it's not in export} und die Sprache ist ähnlich zu Java,
welche grundsätzlich an der Hochschule verwendet wird. \todo{source}
\todo{ist persönliche Meinung okay hier?}
das WIPRO Team war zusätzlich bereits vertraut mit der C\# Sprache und hat diese als angenehm empfunden, was die Entscheidung zusätzlich unterstützte.

\subsubsection*{Entscheid Login Funktionsweise}
\todo{source}
Beim Login wurde überlegt das Switch-Login der Hochschule zu verwenden, welches auch für aussenstehende Zwecke verwendet werden kann.
Hierbei wurde dagegen entschieden, da STAIR nicht die volle Kontrolle über den Service hat und deshalb plötzlich neue Interface Standards erlassen werden können, was zu Problemen führen kann.
Auch muss hierzu ein bestimmter Link geöffnet werden, welcher von Stan verschickt werden muss um die Switch Login Seite zu öffnen.
Dies könnte nicht vertrauenswürdig wirken bei Studenten.
Ausserdem hat sich das Login mittels E-Mails bereits bewährt, weshalb dieses nun weiterhin verwendet wird.

\subsubsection*{Wechsel zu Community Server}
\todo{Yannis}

\subsubsection*{Entscheid zur Anbindung an eine Datenbank}

Eine Datenbank bringt einige Vorteile.
Speziell relevant sind folgende Punkte für dieses Projekt:
\begin{itemize}
    \item Eine Datenbank ermöglicht eine automatische Zuweisung von Studenten zu ihren Häusern.
    \item Sie ermöglicht das automatische markieren von Exstudierenden.
    \item Sie bietet die grösste Erweiterbarkeit für die Zukunft.
    \item Modul-Channels welche zur Zeit nicht gebruacht werden, könnten theoretisch auf Discord gelöscht werden und neu erstellt werden, wenn diese gebraucht werden.
    \item Bei Problemen kann schnell herausgefunden werden, welche genau Person betroffen ist und kann sich auch über alternative Kommunikationswege, wie E-Mail, mit der Person in Kontakt setzen.
    \item Studenten können über ihre E-Mail permanent gebannt werden, ohne dass sie sich mit neuen Discord Accounts wieder anmelden können.
\end{itemize}

Eine Datei wäre ebenfalls möglich zur Datenspeicherung.
Diese hätte jedoch weniger Performance und würde keine Checks durchführen beim Speichern der Daten.
Das Auslesen von gefilterten Daten gestaltet sich ebenfalls einfacher.\autocite{castro_why_2020}
Da bei diesem Projekt wichtig ist, dass die Daten korrekt im Schmea gespeichert werden und die Anpassbarkeit der Datenbank keinen Fokus bekommen hat, wurde eine SQL basierte Datenbank bevorzugt gegenüber von NoSQL Datenbanken.
Als gewählte Technologie hat sich MySQL durch seine grosse Verbreitung sehr gut angeboten.
Online wird MySQL als die zweitmeist genutzte Datenbank-Technologie beschrieben.\autocite{noauthor_db-engines_2022}

\subsubsection*{Entscheidung einen eigenen Bot zu entwickeln}
Zusammen mit STAIR wurde entschieden, wieder eine eigensentwickelte Lösung zu wählen.
Dies hat verschiedene Vorteile.
Als erstes ist damit gleich auch der Datenschutz gewährleistet.
Auch sind so keine künstliche Limitierungen eines externen entwickelten Bots vorhanden.
Die einzigen Limitierungen sind vom API von Discord und von technischen Know-How der betreuenden Entwicklern.
Da die verschiedenen Bots eventuell nicht in Europa gehostet sind, könnten Verbindungs- und/oder Performanceprobleme auftreten.
Durch die Log Datei(en) können Fehler einfacher gefunden und nachvollzogen werden.

\subsection{Initialisierungsphase}
Aus dem Projektauftrag und aus den Wünschen der Stakeholder werden Anforderungen an das Produkt definiert.
Diese werden für die Realisierungsphase in Epics umgewandelt.
Die Epics werden im Anschluss in einzelne User Stories aufgeteilt, die dann in den Sprints seperat behandelt und gelöst werden.

\subsubsection{Epics}\label{Epics}
Für das Projekt werden folgende Epics definiert.

\begin{table}[h]
    \centering
    \begin{tabular}{ | p{1em} | p{35em} | p{2em} |}
        \hline
        \rowcolor[gray]{.9} ID & Beschreibung & Prio \\
        \hline
        1 & Der Nutzer wird beim erstmaligen Betreten des Discord-Servers vom Bot benachrichtigt.
        Dieser gibt ihm Grundlegende Informationen zum Server und dem Authentisierungsprozess.
        Zu diesem Zeitpunkt hat der Student noch keine Berechtigungen auf dem Server und
        hat nur Zugriff auf den Channel "Help". & A \\
        \hline
        2 & Der Bot soll den Nutzer authentifizieren und mit der Studenten-Rolle versehen können.
        Er kann dabei zwischen den E-Mails von Nicht-Student und Student unterscheiden und auf unvorhergesehene
        Interaktionen, von Seiten des Benutzers, reagieren können. & A \\
        \hline
        3 & Ein Student kann sich beim Bot für ein Modul anmelden. Dieser schaltet den Channel für den Studenten frei,
        so das er darin mit anderen Studierenden kommunizieren kann. Falls das Modul für den Studenten nicht mehr relevant ist,
        kann er es beim Bot wieder abmelden. & A \\
        \hline
        4 & Die Administration des Discord Servers soll einfach neue Modullisten in das System laden können.
        Die neuen oder nicht mehr Verfügbaren Module werden erkannt und entsprechend hinzugefügt oder gelöscht.
        An dem Verhalten des Benutzers soll sich nichts ändern. & A \\
        \hline
        5 & Die Adminsitration kann jedes Semster die neuen Studierenden hinzufügen.
        Diese werden beim potentiellen Authentifizieren direkt in Ihre zugeteilten Häuser-Channels zugewiesen. & A \\
        \hline
        6 & Die Administration von STAIR kann über eine zur Verfügung gestellte Schnittstelle, Statistiken über
        den Discord erstellen. & B \\
        \hline
        7 & Fehlereingaben in dem Modulregistrierungschannel oder interne Fehler vom Bot, sollen direkt automatisch
        an einen hinterlegten Mitarbeiter von STAIR gemeldet werden & C \\
        \hline
    \end{tabular}
    \caption{Epics}
    \label{tab: Epics}
\end{table}

\subsubsection{User Stories}
Die Epics werden in kleinere User-Stories heruntergebrochen und dienen als Bausteine eines Sprints.
Dieser Prozess wird auch \textbf{Story-Splitting} genannt
User-Stories sind immer nach dem gleichen Muster aufgebaut. \\
"Als <Rolle> möchte ich <Ziel/Wunsch>, um <Nutzen>." \\
Die Story muss nicht immer aus der Benutzersicht definiert sein, sondern es können auch technische oder
infrastrukturelle User-Stories so gebildet werden.

In der Strintplannung werden zu jeder User-Story zugehörige Akzeptanzkriterien definiert.
Diese dienen zur Kontrolle, ob die User-Story korrekt umgesetzt wurde und
können auch als Test während der Entwicklung gebraucht werden.
Später können dann die Mitglieder des Teams, den Aufwand der Story realistisch abschätzen.

Gute User-Stories können nach dem Modell \textbf{I.N.V.E.S.T} gebildet werden. \autocite{hammerschall_software_2013} % PMB Projektplanung p.101-103
\begin{itemize}
    \item \textbf{I}ndependent  unabhängig voneinander
    \item \textbf{N}egotiable   zerlegbar, änderbar, kombinierbar, verhandelbar
    \item \textbf{V}aluable     hat wirtschaftlichen Wert
    \item \textbf{E}stimatable  so klar, dass es vom Team geschätzt werden kann
    \item \textbf{S}mall        klein genug, um in einem Sprint entwicklet werden zu können
    \item \textbf{T}estable     klare Akzeptanzkriterien
\end{itemize}

Beim Story-Splitting werden folgende User-Stories für das Projekt definiert.

\begin{longtable}{ | p{1em} | p{16em} | p{13em} | p{2em} | p{3em} | p{2em} |}
    \hline
    \rowcolor[gray]{.9} ID & Beschreibung & Akzeptanzkriterien & Prio & Weight & Sprint \\
    \hline
    1 & Als Bot möchte ich den Benutzer beim ersten Betreten des Server authentifizieren können,
    um ihm die ihm zugeteilten Rollen zuweisen zu können. &
    \tabitem Der Nutzer bekommt beim ersten Betreten des Servers eine Nachricht vom Bot mit Informationen. & A & 12 & 2 \\*
    & & \tabitem Der Student bekommt nach der Anmeldung die Rolle Student zugeteilt und wird dem richtigen Haus-Channel zugewiesen. & & & \\
    \hline
    2 & Als Bot möchte ich den Studenten mittels E-Mail Verifikation authentifizieren können,
    um sicher zu gehen, dass dieser eine gültige Studenten-Mail besitzt. &
    \tabitem Der Student bekommt eine E-Mail auf der Adresse, die er dem Bot mitteilt. & A & 15 & 3 \\*
     & & \tabitem Ein Student kann sich authentifizieren, ein Nicht-Student wird abgewiesen. & & & \\*
     & & \tabitem In der E-Mail steht ein zufällig generierter 6-stelliger Code. & & & \\*
     & & \tabitem Der Bot kann den Code verifizieren, nachdem der Student die E-Mail erhalten hat. & & & \\
    \hline
    3 & Als STAIR Administrator möchte ich neue Studenten einfach erfassen können,
    um sie dem System bekannt zu machen und die Administration zu vereinfachen. &
    \tabitem Neue Studenten können mit ihren zugeteilten Häusern als Liste eingelesen werden & B & 9 & 1 \\*
     & & \tabitem Die neu erfassten Studenten werden im System eingetragen und können für die Authentifizierung verwendet werden. & & & \\
    \hline
    4 & Als Student möchte ich mich für neue Modul-Channels an und abmelden können,
    um mich mit anderen Studenten austauschen zu können. &
    \tabitem Es stehen Befehle im \textbf{Registrierungs-Channel} zur Verfügung, um sich bei Modulen an- und abzumelden. & A & 12 & 3 \\*
     & & \tabitem Der Bot weist den Studenten, bei einer Anmeldung, dem Modul als Mitglied hinzu. & & & \\*
     & & \tabitem Der Bot löscht den Studenten, bei einer Abmeldung, aus der Mitgliederliste,
     so dass der Channel für den Studenten nicht mehr sichtbar ist. & & & \\
    \hline
    5 & Als STAIR Administrator möchte ich Modullisten mit den verfügbaren Modulen, einfach in das System eintragen zu können,
    um sie für den Bot nutzbar zu machen. &
    \tabitem Pro Semester können Modullisten automatisch eingelesen werden. & B & 8 & 2 \\*
     & & \tabitem Neue Module werden automatisch hinzugefügt, und solche, welche nicht mehr Unterrichtet werden, werden gelöscht. & & & \\*
     & & \tabitem Die Module stehen nach dem einlesen direkt bereit, zur Channel-Erstellung durch den Bot. & & & \\
    \hline
    6 & Als STAIR Administrator möchte ich Statistiken aus dem System auslesen können,
    um sie für Marketing oder andere Zwecke gebrauchen zu können. &
    \tabitem Es stehen ein paar vorgefertigte Auslesemöglichkeiten bereit, um sie für den Administrator nutzbar zu machen & C & 9 & 4 \\
    \hline
    7 & Als STAIR Administrator möchte ich sofort benachrichtigt werden, falls mit dem System ewtwas nicht stimmt,
    um Gegenmassnahmen ergreifen zu können. &
    \tabitem Der Administrator wird benachrichtigt, wenn der Bot zu Discord oder zur Datenbank keine Verbindung mehr hat & C & 5 & 5 \\*
     & & \tabitem Der Administrator wird benachrichtigt wenn der E-Mail Versand zur Authentifikation nicht mehr funktioniert. & & & \\
    \hline
    \caption{User-Stories}
    \label{tab: UserStories}
\end{longtable}
\subsubsection*{Sprintplanung}
\begin{table}[h]
    \centering
    \begin{tabular}{|l|l|l|l|}
        \hline
        \rowcolor[gray]{.9} Sprint & Start & Ende & Artefakte \\
        \hline
        Sprint 1 & 10.10.2022 & 16.10.2022 & \tabitem Studentenerfassung implementiert \\
         & & & \tabitem Sprintreview S01 \\
         & & & \tabitem Sprintplannung S02 \\
        \hline
        Sprint 2 & 17.10.2022 & 30.10.2022 & \tabitem Modulerfassung implementiert \\
         & & & \tabitem Sprintreview S02 \\
         & & & \tabitem Sprintplannung S03 \\
        \hline
        Sprint 3 & 31.10.2022 & 13.11.2022 & \tabitem Authentifizierung implementiert \\
         & & & \tabitem Sprintreview S03 \\
         & & & \tabitem Sprintplannung S04 \\
        \hline
        Sprint 4 & 14.11.2022 & 27.11.2022 & \tabitem Statistiken implementiert \\
         & & & \tabitem Sprintreview S04 \\
         & & & \tabitem Sprintplannung S05 \\
        \hline
        Sprint 5 & 28.11.2022 & 11.12.2022 & \tabitem Fehlererkennung implementiert \\
         & & & \tabitem Sprintreview S05 \\
        \hline
    \end{tabular}
    \caption{Sprintplanung}
    \label{tab: Sprintplanung}
\end{table}

Die Projektkontrolle und der Fortschritt wird mit folgenden Tools überwacht.
\begin{itemize}
    \item Github Story-Board (\href{https://github.com/orgs/stairch/projects/1/views/4}{Link zum Story-Board})
    \item Sprintreviews (\nameref{Sprintreviews})
\end{itemize}

\subsubsection{Testdrehbuch}\label{Testdrehbuch}
Wie im Kapitel \nameref{Teststrategie} erwähnt, werden im Folgenden die einzelnen Systemtests beschrieben.
Diese müssen alle manuell vor der Übergabe der Software durchgeführt werden.
\newline
Sie beziehen sich immer auf eine Anforderung / Epic und können so auch bei den einzelnen Sprintreviews zur Kontrolle verwendet werden.
Die Referenzierten Epics sind im Kapitel \nameref{Epics} beschrieben.

\begin{longtable}[h]{|p{15em}|p{25em}|}
    \hline
    \multicolumn{2}{|l|}{\textbf{Betritt ein neuer Student den Discord-Server, bekommt er direkt eine Nachricht.}} \\*
    \multicolumn{2}{|l|}{\textbf{vom Stan Bot.}} \\
    \hline
    \multicolumn{2}{|l|}{\textbf{Testfall \#1}} \\
    \hline
    \textbf{Epic / Anforderung} & \#1 \\
     & Der Nutzer wird beim erstmaligen Betreten des Discord-Servers vom Bot benachrichtigt.
     Dieser gibt ihm Grundlegende Informationen zum Server und dem Authentifizierungsprozess.
     Zu diesem Zeitpunkt hat der Student noch keine Berechtigungen auf dem Server und
     hat nur Zugriff auf den Channel "Help". \\
    \hline
    \textbf{Durchführung} &
    \begin{enumerate}
        \item Bereitstellen eines Discord Accounts, welcher noch nicht auf dem STAIR-Discord Server angemeldet ist.
        \item Dem STAIR Discord Server beitreten. \url{https://discord.com/invite/Tp7XgzZ}
    \end{enumerate}\\
    \hline
    \textbf{Erwartetes Ergebnis / Verhalten} & Sobald man dem Server beigetreten ist, bekommt man vom Stan Bot eine private Nachricht.
    Dort stehen allgemeine Informationen zum Server und zum Anmeldeprozess. \\
    \hline
    \caption{Testdrehbuch - Testfall \#1}
\end{longtable}

\begin{longtable}[h]{|p{15em}|p{25em}|}
    \hline
    \multicolumn{2}{|l|}{\textbf{Ein Student kann sich beim Bot authentifizieren.}} \\
    \hline
    \multicolumn{2}{|l|}{\textbf{Testfall \#2}} \\
    \hline
    \textbf{Epic / Anforderung} & \#2 \\*
     & Der Bot soll den Nutzer authentifizieren und mit der Studenten-Rolle versehen können. \\
    \hline
    \textbf{Durchführung} &
    \begin{enumerate}
        \item Der Student hat eine Nachricht von Stan als Direkt Nachricht erhalten.
        \item Der Student schickt dem Bot seine E-Mail Adresse nach dem Muster \textit{<name.vorname>@stud.hslu.ch} .
        \item Nach Erhalt der E-Mail, schickt der Student, Stan den 6-stelligen Verifizierungscode.
    \end{enumerate}\\
    \hline
    \textbf{Erwartetes Ergebnis / Verhalten} & Der Student sollte nach Senden seiner E-Mail Adresse dem Bot, eine E-Mail von ihm erhalten.
    In dieser ist ein zufällig generierter 6-stelliger Verifizierungscode enthalten.
    Nach dem verifizieren, bekommt der Nutzer die Rolle Student und "Haus" zugeteilt und hat Zugriff auf die Grund-Channels. \\
    \hline
    \caption{Testdrehbuch - Testfall \#2}
\end{longtable}

\begin{longtable}[h]{|p{15em}|p{25em}|}
    \hline
    \multicolumn{2}{|l|}{\textbf{Ein Nicht-Student kann sich nicht authentifizieren.}} \\
    \hline
    \multicolumn{2}{|l|}{\textbf{Testfall \#3}} \\
    \hline
    \textbf{Epic / Anforderung} & \#2 \\*
     & Er kann dabei zwischen den E-Mails von Nicht-Student und Student unterscheiden und
     auf unvorhergesehene Interaktionen, von Seiten des Benutzers, reagieren können. \\
    \hline
    \textbf{Durchführung} &
    \begin{enumerate}
        \item Ein Nicht-Student schickt dem Bot eine E-Mail, welche sich vom Muster \textit{<name.vorname>@stud.hslu.ch} unterscheidet.
    \end{enumerate}\\
    \hline
    \textbf{Erwartetes Ergebnis / Verhalten} & Der Nutzer bekommt direkt eine Nachricht vom Bot, das diese E-Mail nicht gültig ist.
    Es können sich nur Studenten authentifizieren. \\
    \hline
    \caption{Testdrehbuch - Testfall \#3}
\end{longtable}

\begin{longtable}[h]{|p{15em}|p{25em}|}
    \hline
    \multicolumn{2}{|l|}{\textbf{Ein STAIR Administartor kann eine Liste mit Studenten in das System einlesen.}} \\
    \hline
    \multicolumn{2}{|l|}{\textbf{Testfall \#4}} \\
    \hline
    \textbf{Epic / Anforderung} & \#5 \\*
     & Die Adminsitration kann jedes Semester die neuen Studierenden hinzufügen.
     Diese werden beim potentiellen Authentifizieren direkt in Ihre zugeteilten Häuser-Channels zugewiesen. \\
    \hline
    \textbf{Durchführung} &
    \begin{enumerate}
        \item Der STAIR Administrator muss sich eine Liste von Studenten mit zugehörigen Häusern von der Hochschul-Administration besorgen.
        \item Er kann die Liste als Parameter an ein "LoadStudent" Script übergeben.
    \end{enumerate}\\
    \hline
    \textbf{Erwartetes Ergebnis / Verhalten} & Die Liste wird automatisch in das System eingelesen und die entsprechenden Datenbankeinträge werden erstellt.
    Studenten, welche das letzte Semester bestanden haben und nicht mehr auf der Liste vorhanden sind, werden als Ex-Studenten markiert.\\
    \hline
    \caption{Testdrehbuch - Testfall \#4}
\end{longtable}

\begin{longtable}[h]{|p{15em}|p{25em}|}
    \hline
    \multicolumn{2}{|l|}{\textbf{Ein STAIR Administrator kann eine Liste mit Modulen in das System einlesen.}} \\
    \hline
    \multicolumn{2}{|l|}{\textbf{Testfall \#5}} \\
    \hline
    \textbf{Epic / Anforderung} & \#4 \\*
     & Die Administration des Discord Servers soll einfach neue Modullisten in das System laden können.
     Die neuen oder nicht mehr verfügbaren Module werden erkannt und entsprechend hinzugefügt oder gelöscht.
     An dem Verhalten des Benutzers soll sich nichts ändern. \\
    \hline
    \textbf{Durchführung} &
    \begin{enumerate}
        \item Der STAIR Administrator muss sich eine Liste mit allen verfügbaren Modulen des Departements Informatik von der Hochschul-Administration besorgen.
        \item Er kann die Liste als Parameter an ein "LoadModules" Script übergeben.
    \end{enumerate}\\
    \hline
    \textbf{Erwartetes Ergebnis / Verhalten} & Alle Module sollen in das System eingelesen werden und
    die entsprechenden Datenbankeinträge werden erstellt.
    Neue Module werden automatisch hinzugefügt.
    Module, welche nicht mehr auf der Liste vorhanden sind, werden gelöscht. \\
    \hline
    \caption{Testdrehbuch - Testfall \#5}
\end{longtable}

\begin{longtable}[h]{|p{15em}|p{25em}|}
    \hline
    \multicolumn{2}{|l|}{\textbf{Ein Student kann sich beim Bot für ein Modul anmelden.}} \\
    \hline
    \multicolumn{2}{|l|}{\textbf{Testfall \#6}} \\
    \hline
    \textbf{Epic / Anforderung} & \#3 \\*
     & Ein Student kann sich beim Bot für ein Modul anmelden.
     Dieser schaltet den Channel für den Studenten frei, so, dass er darin mit anderen Studierenden kommunizieren kann. \\
    \hline
    \textbf{Durchführung} &
    \begin{enumerate}
        \item Der Student geht auf den Registrierungs-Channel.
        \item Er schreibt \textit{show <module>} in den Chat. (Also z.B. \textit{show ISF})
    \end{enumerate}\\
    \hline
    \textbf{Erwartetes Ergebnis / Verhalten} & Der Bot erkennt das Modul und fügt den Studenten automatisch als Mitglied diesem Text-Channel hinzu.
    Der Channel ist nun für den Studenten sichtbar. \\
    \hline
    \caption{Testdrehbuch - Testfall \#6}
\end{longtable}

\begin{longtable}[h]{|p{15em}|p{25em}|}
    \hline
    \multicolumn{2}{|l|}{\textbf{Ein Student kann sich beim Bot bei einem Modul abmelden.}} \\
    \hline
    \multicolumn{2}{|l|}{\textbf{Testfall \#7}} \\
    \hline
    \textbf{Epic / Anforderung} & \#3 \\*
     & Falls das Modul für den Studenten nicht mehr relevant ist, kann er es beim Bot wieder abmelden. \\
    \hline
    \textbf{Durchführung} &
    \begin{enumerate}
        \item Der Student geht auf den Registrierungs-Channel.
        \item Er schreibt \textit{hide <module>} in den Chat. (Also z.B. \textit{hide ISF})
    \end{enumerate}\\
    \hline
    \textbf{Erwartetes Ergebnis / Verhalten} & Der Bot erkennt das Modul und meldet den Studenten automatisch davon ab.
    Dies entfernt ihn von der Mitgliederliste des Channels und wird dadurch für ihn wieder nicht einsehbar. \\
    \hline
    \caption{Testdrehbuch - Testfall \#7}
\end{longtable}

\begin{longtable}[h]{|p{15em}|p{25em}|}
    \hline
    \multicolumn{2}{|l|}{\textbf{Ein STAIR Administrator kann über ein Script Statistiken auslesen.}} \\
    \hline
    \multicolumn{2}{|l|}{\textbf{Testfall \#8}} \\
    \hline
    \textbf{Epic / Anforderung} & \#6 \\*
     & Die Administration von x kann über eine zur Verfügung gestellte Schnittstelle, Statistiken über den Discord erstellen. \\
    \hline
    \textbf{Durchführung} &
    \begin{enumerate}
        \item Der STAIR Administrator kann sich mit dem Script "Statistics" über die verschiedenen Statistiken informieren.
        \item Er kann einen Befehl auf diesem Script ausführen.
    \end{enumerate}\\
    \hline
    \textbf{Erwartetes Ergebnis / Verhalten} & Je nach Befehl gibt das Script die gewünschten Werte zurück.
    So können zum Beispiel die Anzahl Studenten pro Haus, oder die momentanen Modulanmeldungen ausgelesen werden. \\
    \hline
    \caption{Testdrehbuch - Testfall \#8}
\end{longtable}
\clearpage

\subsection{Konfigurationsmanagement C\# Solution}
\todo{Nicolas}

\begin{table}[h]
    \centering
    \begin{tabular}{|l|p{20em}|l|}
        \hline
        \rowcolor[gray]{.9} Name & Beschreibung & Version \\
        \hline
        .NET Core & Core Library für C\# Projekte.
        Enthält alle grundlegend Klassen und Libraries, welche man zum Arbeiten braucht. & 6.0.0 \\
        \hline
    \end{tabular}
    \caption{Konfigurationsmanagement}
    \label{tab: Konfigurationsmanagement}
\end{table}
\todo{Welches Projekt enthält welche Libraries}

\subsection{Sprint 1}
Zum Start von Sprint 1 sieht der aktuelle Sprint Plan mit den User-Stories wie folgt aus.
\begin{figure}[h]
    \centering
    \hspace*{-2cm}
    \includegraphics[width=1.3\textwidth]{img/Start_Sprint1_Stories.jpg}
    \caption{Start Sprint 1 User Stories}
    \label{fig:start_sprint_one}
\end{figure}

In der Abbildung erkennt man, dass sich Sprint 1 mit der Studentenerfassung und
mit dem Aufsetzten der Linux Umgebung befasst.

\subsubsection{Entity Relationship Diagramm}
Um die Software Architektur richtig planen zu können, muss im Vorfeld festgelegt werden,
welche Daten im System wichtig sind und wie diese gespeichert werden.
Um diesen Zusammenhang richtig darstellen zu können, wird ein Entity-Relationship Diagramm (ERD) erstellt.
Das ERD kann später in ein Datenbank Schema übertragen werden.
Die einzelnen Tabellen sind wie folgt beschrieben.
\todo{Einfügen ER-Diagramm}
\newline
Bei der Datenspeicherung wird eine Unterteilung zwischen dem Studenten und seinem Discord Account vorgenommen.
Dies erlaubt es, eine saubere Trennung zwischen den richtigen Studenten und ihrem Online erscheinen auf Discord zu gewährleisten.
(Seperation of Concerns (SOC)).
Die Studentendaten, wie Name, Vorname, E-Mail Adresse und Hauszuteilung, werden von der Hochschul-Administration zur Verfügung gestellt und können als Liste eingelesen werden.
Dabei werden die Tabellen \textit{Student} und \textit{House} aufgefüllt und die Zuordnung gemacht.
Implementiert wird die Funktionalität in diesem Sprint 1 (\nameref{tab: UserStories}).
\newline
Der \textit{DiscordAccount} wird bei der Authentifikation dem Studenten zugeordnet.
Über die Tabelle \textit{AccountRole}, werden nach dem Authentifizierungsprozess dem Studenten, die richtigen Rollen zugeordnet.
\newline
Die Tabelle \textit{Module} wird auch über eine, von der Hochschul-Administration, zur Verfügung gestellte Liste befüllt.
Wenn nun der Student sich im Discord beim Bot für ein Modul anmeldet, wird die Zuordnung von Modul und DiscordAccount vorgenommen.
Der richtige Student, bzw. seine Daten, haben mit dieser Zuteilung nichts zu tun.
\newline
Die Tabelle \textit{Category} ist für die technische Umsetzung der Modulgestaltung in Discord möglich.
Beschrieben wird dieses Problem im Kapitel \nameref{discord_einschraenkungen}.

\subsection{Aufsetzten Solution}
Die C\# Solution wird in drei Projekte aufgeteilt, welche jeweils andere Verantwortlichkeiten haben.
\begin{itemize}
    \item StanBot
    \item StanDatabase
    \item StanScripts
\end{itemize}

In dem \textbf{StanBot} Projekt wird der Discord Bot erstellt und alle Funktionalität liegen,
die mit dem Bot zusammenhängt.\\
In dem \textbf{StanDatabase} Projekt wird die Datenbank Anbindung verwaltet.
Darin werden die Tabellen der Datenbank auf zugeweisene Klassen gemappt.
Mithilfe der Library LinQ2DB werden die Tabellen der Datenbank, dann über diese Klassen verwaltet.
Datenbank Operationen sollen nur in diesem Projekt erfolgen. 
Die anderen Projekte sollen keinen direkten Zugriff auf die Datenbank haben.\\
Das \textbf{StanScripts} Projekt wird alle Scripte enthalten, 
die für die Verwaltung des Discord Servers und der Datenbank nötig sind.
Darin werden sich die Scripte für die Studenten -und Modulerfassung befinden,
sowie für die Statistiken.

\subsubsection{Repository Pattern für die Daten}
Wie im vorherigen Kapitel erwähnt, soll nur das StanDatabase Projekt direkten Zugriff auf die Datenbank haben.
Wenn der Discord Bot, oder ein Script, Datenbankoperationen ausführen will, wird dazu das Repository Pattern verwendet.\\
Das Repository Pattern ist dazu da, die Create, Read, Update und Delete (CRUD) Funktionen, 
welche auf Tabellen durchgeführt werden, zu kapseln. \autocite{}
%https://offering.solutions/blog/articles/2014/10/06/asp.net-repository-pattern-und-unit-of-work/#:~:text=Repositories%3A,die%20CRUD%2DOperationen%20ausf%C3%BChren%20kann.
Es sieht vor, dass jedes Objekt genau eine Schnittstelle hat, über die Veränderung daran vorgenommen werden können.

\begin{figure}[h]
    \centering
    \includegraphics[width=1\textwidth,height=5cm]{img/Repository_Pattern.png}
    \caption{Darstellung Repository Pattern}
    \label{fig:repository_pattern}
\end{figure}
%https://www.google.ch/url?sa=i&url=https%3A%2F%2Fmedium.com%2Fcorebuild-software%2Fandroid-repository-pattern-using-rx-room-bac6c65d7385&psig=AOvVaw3r0fNPN8XItp0pPq7KFvUP&ust=1668526531304000&source=images&cd=vfe&ved=0CBAQjRxqFwoTCMiumpaDrvsCFQAAAAAdAAAAABAD

Wie in der Darstellung zu erkennen, gibt es für andere Klassen nur noch einen Zugriffspunkt für das gewünschte Datenobjekt.
\newline
Ein weiterer Vorteil dieses Patterns ist die vereinfachte Testbarkeit.
Da das Repository selbst eine Schnittstelle, also ein Interface ist, kann es meherere Implementationen davon geben.
Für die richtige Applikation kann nun eine Implementation erstellt werden, die wirklich Daten auf der Datenbank verändert,
und für das Testen kann eine Implementation erstellen werden, die nur Mock Daten enthält.\\
Mithilfe von Dependency Injection kann nun jeweils die gewünschte Implementation für das Repository übergeben werden.

\begin{lstlisting}[language=csharp]
// IHouseRepository
public interface IHouseRepository
{
    House GetHouseByName(string houseName);
    bool IsHouseNameValid(string houseName);
}


// HouseRepositoryImpl
public class HouseRepository : IHouseRepository
{
    public House GetHouseByName(string houseName)
    {
        using (var db = new DbStan()) { ... }
    }

    public bool IsHouseNameValid(string houseName)
    {
        using (var db = new DbStan()) { ... }
    }
}

// LoadStudentScript
public class LoadStudents
{
    private readonly IHouseRepository _houseRepository;
    public LoadStudents(IHouseRepository houseRepository)
    {
        _houseRepository = houseRepository;
    }

    public void LoadStudentsFromFile(string filePath)
    {
        ...
        _houseRepository.GetHouseByName(values[houseIndex]);
        ...
    }
}

// Program
public static class Program
{
    public static void Main(string[] args)
    {
        LoadStudents loadStudents = new LoadStudents(new HouseRepository());
        loadStudents.LoadStudentsFromFile(args[1]);
    }
}
\end{lstlisting}

\subsubsection{Studentenerfassung}

Die Studenten werden über ein vordefiniertes CSV File geladen.
Hier wird ein Beispiel dieser Datei gezeigt:

\todo{Add CSV data here}

\todo{Yannis}

\begin{lstlisting}[language=csv]

\end{lstlisting}

\subsubsection{Aufsetzten der Linux Umgebung}

Der Ubuntu Server wurde vom Enterprise Lab aufgesetzt.
Diese musste 

\todo{Siehe Kapitel Anleitung}

\todo{Yannis}

\newpage
\subsection{Sprint 2}
In diesem Sprint wird sich mit der Modulerfassung per Script beschäftigt und dem Erstellen des Discord Bots.

\subsubsection{Modulerfassung}
\todo{Yannis}

\subsubsection{Aufsetzten Discord Bot}
\todo{Discord Developer Page: Applikationen im Web.}

\subsubsection{Stan Bot Architektur}
Grundlegend geht es bei der Bot Architektur um die Kommunikation mit der DiscordClient Klasse.
Diese dient als Client-Komponente des Bots und man kann sich dort auf alle Events abonnieren, die Discord anbietet.\\

\subsubsection*{Services und Dependency Injection}
In der Main-Methode der Applikation wird ein Host Environment für die Applikation erstellt.
Dieses Environment braucht man, damit der Bot später als Deamon auf der Linux Umgebung im Hintergrund laufen kann.
Des weiteren bietet es noch Funktionen zur Dependency Injection an.
Alle Klassen können dort als Singleton, Scoped oder Transient registriert werden.
Aus der Microsoft .NET Dokumentation:
\textit{
\begin{itemize}
    \item Transient-Vorgänge sind immer unterschiedliche, und mit jedem Abruf des Diensts wird eine neue Instanz erstellt.
    \item Scoped-Vorgänge ändern sich nur durch einen neuen Bereich, die Instanz ist jedoch innerhalb eines Bereichs immer gleich.
    \item Singleton-Vorgänge sind immer gleich, und eine neue Instanz wird nur einmal erstellt. \autocite{}
\end{itemize}
}
%https://learn.microsoft.com/de-de/dotnet/core/extensions/dependency-injection-usage#conclusion

\begin{lstlisting}[language=csharp]
// Program.cs
using IHost host = Host.CreateDefaultBuilder()
            .ConfigureServices((_, services) => services
                .AddSingleton(new DiscordSocketClient())
                .AddSingleton<EventHandler>()
                .AddScoped<IStudentRepository, StudentRepository>()
            .Build();
\end{lstlisting}

In der Bot Klasse kann nun mithilfe eines Providers des Host Environments, die Instanz des gebrauchten Services geholt werden.

\begin{lstlisting}[language=csharp]
// Bot.cs
using IServiceScope serviceScope = _hostEnvironment.Services.CreateScope();
IServiceProvider provider = serviceScope.ServiceProvider;
_discordSocketClient = provider.GetRequiredService<DiscordSocketClient>();
\end{lstlisting}

Die Klasse kann aber auch direkt in einen Konstruktor \textit{Injected} werden.
Auf diese Weise hat man an einer Stelle die Kontrolle über alle verfügbaren Klassen und Instanzen.
\begin{lstlisting}[language=csharp]
// EventHandler.cs
public class EventHandler
{
    private readonly DiscordSocketClient _discordSocketClient;
    public EventHandler(DiscordSocketClient discordSocketClient)
    {
        _discordSocketClient = discordSocketClient;
    }
}
\end{lstlisting}

\subsubsection*{Laden der Config}
Beim Starten des Discord Bots muss eine Konfiguration mit spezifischen \textbf{DiscordApplicationToken} mitgegeben werden.
Dieses Application Token ist wie der Private Key der Applikation und sollte nicht veröffentlicht werden.
Um diese und weitere sensible Informationen einlesen zu können, wird die Config Klasse erstellt.\\
Diese enthält alle Felder der Settingsdatei und dient als Zugriffspunkt für die sensiblen Informationen.
Sie liest mit einem einfachen JSON-Parser die Settings Datei ein und setzt ihre Attribute.

Die Settings Datei heisst \textit{stan.json}
\begin{lstlisting}[language=json]
{
    "DisordApplicationToken": "XXXX-XXXXX-XXXXXXXXXXXX",
    "Prefix": "!"
}
\end{lstlisting}

\subsubsection{Kommunikation mit Discord}

\newpage
\subsection{Sprint 3}

\subsubsection{Command Service Architektur}
\todo{Nicolas}

\subsubsection{show/hide <module> Command}

Die beiden Commands show und hide können von Studenten benutzt werden um sich selbst an einem Modul Discord Channel hinzuzufügen und sich dann über das Modul austauschen zu können.
Der Name des Moduls muss hierfür spezifisch formatiert sein.
Eine Anleitung hierzu wird dem Studenten auf dem Channel zum Anmelden bei den Modulen angezeigt.
Diese Anleitung wird auch gepinnt, damit sie zwischen den vielen Nachrichten nicht untergeht.

Der angegebene Modulname muss dem Kurznamen entsprechen.
Gross und Kleinschreibung darf ignoriert werden, da im hintergrund dies einheitlich umgewandelt wird.
Einige Module haben am Ende des Kurznamens eine Erweiterung um die verschiedenen Durchführungen, z.B. bei verschiedenen Professoren, zu unterscheiden.
Auch der Prefix und Suffix muss entfernt werden, welche den Studiengang, bzw. das Jahr und Semester beinhaltet.

Beispiele:

\begin{table}[h]
    \centering
    \begin{tabular}{|1|1|1|}
        \hline
        \rowcolor[gray]{.9} Angezeigt im MyCampus & Beispiel Befehl & Voller Modulname \\
        \hline
        I.BA_GAMEDEV.H2201 & show gamedev & Game Development
        \hline
        TODO
    \end{tabular}
    \caption{Show und Hide Command Beispiele}
    \label{tab: Show und Hide Command Beispiele}
\end{table}

Diese Umwandlung muss ebenfalls beim Laden der Modul-Liste des Sekretariats durchgeführt werden.
Diese automatische Umwandlung wurde durch Unit Tests getestet. \todo{verlinke Testkapitel}
\todo{Yannis}
\todo{add image of mycampus calendar}

\subsubsection{E-Mail Service}
\todo{Nicolas}

\newpage
\section{Evaluation und Validation}

\todo{schreiben}

\subsection{Fehlererkennung}

\section{Ausblick}

\todo{schreiben}

\newpage

\section{Anh\"ange}

\subsection{Sprintreviews}\label{Sprintreviews}
In diesem Kapitel werden die einzelnen Sprintreviews augelistet.
Diese beinhalten immer eine Reflektion zum letzten Sprint mit einer Übersicht der umgesetzten User-Stories und
einer Retrospektive über aufgefallenen Problemen und dazugehörigen Massnahmen.

\subsubsection{Sprint 1}
\subsubsection*{Sprintziel}
Studentenerfassung ist implementiert.

\subsubsection*{Risiko-Update}
\begin{itemize}
    \item Kommunikation in Richtung Discord wurde im Vorfeld noch nicht bedacht.
    Kann sein das die Architektur angepasst werden muss, um Kommunikation ausgehend vom Bot starten zu können.
\end{itemize}

\subsubsection*{Sprintbacklog}
In diesem Sprint wurde nur eine User-Story umgesetzt.\\
\url{https://github.com/stairch/stan-discord-bot/issues/2}
\newline
Geschätzer Aufwand: 8h
\newline
Effektiver Aufwand: 6 1/2 h

\subsubsection*{Retrospektive}
Erfolge:
\begin{itemize}
    \item LoadStudenten Script konnte einfach und schnell erstellt werden.
    \item Library Linq2db einfacher als erwartet.
\end{itemize}
Probleme:
\begin{itemize}
    \item Linq2db stellt keine Generierung der Datenbank aus den Klassen zur Verfügung.
\end{itemize}
Massnahmen:
\begin{itemize}
    \item Datenbank Schema muss manuell erstellt werden.
\end{itemize}

\subsubsection{Sprint 2}
\todo{Same for Sprint 2}
\newpage

\section{Glossar}

API

CSV

z.B.

bzw.

% https://blog.aristolo.com/de/definition-begriffe-bachelorarbeit-masterarbeit-dissertation/
\todo{schreiben}

\section{Abbildungsverzeichnis}
\listoffigures

\section{Tabellenverzeichnis}
\listoftables

\section{Literaturverzeichnis}
\printbibliography

\section{Bedienungsanleitung}

\subsection{Aufsetzen des Servers}

Der Stan Discord Bot kann auf verschiedenen Betriebssystemen genützt werden.
In dieser Anleitung wird das Aufsetzen für Ubuntu Linux 20.04 gezeigt.

\todo{Complete tutorial}

\begin{enumerate}
    \item Verbinde zum VPN oder verbinde zum WLAN an der Hochschule
    \item Verbinde zum Server über SSH
        \begin{verbatim}
            $ ssh localadmin@stair-bot-lnx.el.eee.intern
        \end{verbatim}
    \item Führe Updates durch
        \begin{verbatim}
            # sudo apt update
            $ apt list --upgradable
            # sudo apt upgrade -y
            # sudo apt autoremove
        \end{verbatim}
    \item Installiere MySQL
        \begin{verbatim}
            # sudo apt install mysql-server
            # sudo systemctl start mysql.service
        \end{verbatim}
    \item Installiere .NET 6
        \begin{verbatim}
            # wget https://packages.microsoft.com/config/ubuntu/20.04/packages-microsoft-prod.deb -O packages-microsoft-prod.deb
            # sudo dpkg -i packages-microsoft-prod.deb
            $ rm packages-microsoft-prod.deb
            # sudo apt update
            # sudo apt install -y dotnet-sdk-6.0
        \end{verbatim}
    \item Lade den Code auf den Server
        \begin{verbatim}
            # 
        \end{verbatim}
    \item Erstelle die Datenbank
        \begin{verbatim}
            # 
        \end{verbatim}
    \item Starte die Software
        \begin{verbatim}
            # sudo apt update
            # sudo apt upgrade -y
        \end{verbatim}
\end{enumerate}

\subsection{Unterhalt}

Einmal pro Semester müssen die Module und die Studierenden aktualisiert werden.
\todo{Write tutorial}

\end{document}
