\documentclass[a4paper]{article}

\usepackage{amssymb} 
\usepackage[a4paper]{geometry}
\usepackage[T1]{fontenc}
\usepackage[utf8]{inputenc}
\usepackage{graphicx}
\graphicspath{ {images/} }
\usepackage{todonotes}
\usepackage{hyperref}
\hypersetup{
    colorlinks=true,
    linkcolor=black,
    filecolor=magenta,
    urlcolor=blue
}
% changes font
\usepackage{times}
% Set name of image label
\renewcommand{\figurename}{Abbildung}

\title{
    {Wirtschaftsprojekt} \\
    \vspace{10mm}
    { Herbstsemester 2022 } \\
    \vspace{10mm}
    % https://commons.wikimedia.org/wiki/File:HSLU_2022_logo.svg
    {\includegraphics[width=75mm]{img/hsluLogo2022.png}}
}

\author{Yannis Krämer und Nicolas Wiedmer}
% TODO: Update date
\date{22.09.2022}

\begin{document}

\maketitle

\newpage

\noindent
\fontsize{12}{14}
\textbf{Wirtschaftsprojekt an der Hochschule Luzern -- Informatik} \\ \vspace*{0.6cm}

\fontsize{10.95}{12}
\noindent
\textbf{Titel:} Stair Discord Bot \\ \vspace*{0.2cm}

\noindent
\textbf{Studentin/Student:} Yannis Krämer \newline \newline
\textbf{Studentin/Student:} Nicolas Wiedmer \newline \newline
\textbf{Studiengang:} BSc Informatik oder Wirtschaftsinformatik  \newline \newline
\textbf{Jahr:} 2022 \newline \newline
\textbf{Betreuungsperson:} Markus Waldmann \newline \newline
\textbf{Expertin/Experte:} \newline \newline
\textbf{Auftraggeberin/Auftraggeber:} Stair (Martin Steiger \& Estefania Otero)\newline \newline \newline
\textbf{Codierung / Klassifizierung der Arbeit:}\\
$\boxtimes$ \"Offentlich 
$\square$ Vertraulich


%%% you can use \boxtimes for filling a cross inside the square
%%% e.g., $\boxtimes$ A: Einsicht 	(Normalfall) 


\paragraph{\textbf{Eidesstattliche Erkl\"arung}}
Ich erkl\"are hiermit, dass ich/wir die vorliegende Arbeit selbst\"andig und ohne unerlaubte fremde Hilfe angefertigt haben, alle verwendeten Quellen, Literatur und andere Hilfsmittel angegeben haben, w\"ortlich oder inhaltlich entnommene Stellen als solche kenntlich gemacht haben, das Vertraulichkeitsinteresse des Auftraggebers wahren und die Urheberrechtsbestimmungen der Hochschule Luzern respektieren werden. \newline \newline
Ort / Datum, Unterschrift	\underline{\hspace*{8cm}} \newline \newline
Ort / Datum, Unterschrift	\underline{\hspace*{8cm}} \newline \newline \newline
\textbf{Ausschliesslich bei Abgabe in gedruckter Form: \\
Eingangsvisum durch das Sekretariat auszuf\"ullen}\newline \newline
Rotkreuz, den	\underline{\hspace*{4cm}}	\hspace*{1cm} Visum:	\underline{\hspace*{4cm}}

\newpage
\section*{I{\hspace*{1cm}}Abstract}
\todo{schreiben}

\newpage

\tableofcontents
\newpage
\section{Problem, Fragestellung, Vision}

\todo{schreiben}

\section{Stand der Technik}

\todo{schreiben}

\subsection{Discord Einschränkungen}

% https://netcord.site/new-discord-channel-creation-ratelimit/
% https://support.discord.com/hc/en-us/community/posts/360056762431-Increase-channel-limit
% https://support.discord.com/hc/en-us/community/posts/360032363631-Increase-the-Max-Number-of-Roles-per-Server

\section{Ideen und Konzepte}

\todo{einfügen}

\section{Methoden}

\todo{einfügen}

\subsection{Technologien}

\todo{einfügen}
% https://www.nuget.org/packages/linq2db.MySql/
% http://www.primaryobjects.com/2009/01/24/using-mysql-and-linq-to-sql-in-c-asp-net/

\section{Realisierung}

\todo{schreiben}

\section{Evaluation und Validation}

\todo{schreiben}

\section{Ausblick}

\todo{schreiben}

\newpage

\section{Anhänge}

\todo{schreiben}

\section{Glossar}

% https://blog.aristolo.com/de/definition-begriffe-bachelorarbeit-masterarbeit-dissertation/
\todo{schreiben}

\section{Abbildungsverzeichnis}
\listoffigures

\section{Tabellenverzeichnis}
\listoftables

\section{Literaturverzeichnis}

\todo{einfügen}

\end{document}
