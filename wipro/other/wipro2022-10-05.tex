\documentclass[a4paper]{article}

\usepackage{amssymb}
\usepackage[a4paper]{geometry}
\usepackage[T1]{fontenc}
\usepackage[utf8]{inputenc}
\usepackage{graphicx}
\graphicspath{ {images/} }
\usepackage{todonotes}
\usepackage{hyperref}
\hypersetup{
    colorlinks=true,
    linkcolor=black,
    filecolor=magenta,
    urlcolor=blue
}
% changes font
\usepackage[style=verbose-ibid,backend=bibtex]{biblatex}
\bibliography{wipro-main-doc}

\usepackage{times}
% Set name of image label
\renewcommand{\figurename}{Abbildung}

\title{
    {Wirtschaftsprojekt} \\
    \vspace{10mm}
    { Herbstsemester 2022 } \\
    \vspace{10mm}
    % https://commons.wikimedia.org/wiki/File:HSLU_2022_logo.svg
    {\includegraphics[width=75mm]{img/hsluLogo2022.png}}
}

\author{Yannis Kr\"amer und Nicolas Wiedmer}
% TODO: Update date
\date{22.09.2022}

\begin{document}

\maketitle

\newpage

\noindent
\fontsize{12}{14}
\textbf{Wirtschaftsprojekt an der Hochschule Luzern -- Informatik} \\ \vspace*{0.6cm}

\fontsize{10.95}{12}
\noindent
\textbf{Titel:} STAIR Discord Bot \\ \vspace*{0.2cm}

\noindent
\textbf{Studentin/Student:} Yannis Kr\"amer \newline \newline
\textbf{Studentin/Student:} Nicolas Wiedmer \newline \newline
\textbf{Studiengang:} BSc Informatik oder Wirtschaftsinformatik  \newline \newline
\textbf{Jahr:} 2022 \newline \newline
\textbf{Betreuungsperson:} Markus Waldmann \newline \newline
\textbf{Expertin/Experte:} \newline \newline
\textbf{Auftraggeberin/Auftraggeber:} STAIR (Martin Steiger \& Estefania Otero)\newline \newline \newline
\textbf{Codierung / Klassifizierung der Arbeit:}\\
$\boxtimes$ \"Offentlich
$\square$ Vertraulich


%%% you can use \boxtimes for filling a cross inside the square
%%% e.g., $\boxtimes$ A: Einsicht 	(Normalfall)


\paragraph{\textbf{Eidesstattliche Erkl\"arung}}
Ich erkl\"are hiermit, dass ich/wir die vorliegende Arbeit selbst\"andig und ohne unerlaubte fremde Hilfe angefertigt haben, alle verwendeten Quellen, Literatur und andere Hilfsmittel angegeben haben, w\"ortlich oder inhaltlich entnommene Stellen als solche kenntlich gemacht haben, das Vertraulichkeitsinteresse des Auftraggebers wahren und die Urheberrechtsbestimmungen der Hochschule Luzern respektieren werden. \newline \newline
Ort / Datum, Unterschrift	\underline{\hspace*{8cm}} \newline \newline
Ort / Datum, Unterschrift	\underline{\hspace*{8cm}} \newline \newline \newline
\textbf{Ausschliesslich bei Abgabe in gedruckter Form: \\
Eingangsvisum durch das Sekretariat auszuf\"ullen}\newline \newline
Rotkreuz, den	\underline{\hspace*{4cm}}	\hspace*{1cm} Visum:	\underline{\hspace*{4cm}}

\newpage
\section*{I{\hspace*{1cm}}Abstract}
\todo{schreiben}

\newpage

\tableofcontents
\newpage
\section{Problem, Fragestellung, Vision}

\todo{schreiben}

\section{Stand der Technik}

\subsection{Sprach- und Text-Messenger}

\subsection{Discord}
Der Onlinedienst Discord ist ein Instant-Messaging und Chat Tool mit dem auch Sprach- und Videokonferenzen
abgehalten werden k\"onnen. Er kann online auf einer Webseite aufgerufen werden, oder mithilfe eines Clients lokal.
Discord unterst\"utzt alle g\"angigen Betriebssysteme und kann auch auf mobilen Endger\"aten verwendet werden.

\subsubsection{Geschichte}
Urspr\"unglich wurde Discord f\"ur Computerspiele geschaffen, um nebenher mit seinen Spielkameraden zu texten oder
miteinander zu reden. Das Ziel war es, neben der komplexen Spielmechanik einen Chat-Messanger zu bauen, der
benutzerfreundlich und effizient ist. 2012 wurde das Unternehmen Discord Inc (damals noch Hammer \& Chisel)
als Startup gegr\"undet.\autocite{} 2014 erhielt Hammer \& Chisel f\"ur die Weiterentwicklung von Discord zus\"atzliche
Finanzierungsmittel von verschiedenen Unternehmen.

2015 wurde Discord unter der Domain discordapp.com ver\"offentlicht. Der Text und Sprach Messanger wurde immer beliebter
und erhielt immer mehr Finanzierungskosten, wie 2016 20 Millionen von WarnerMedia \autocite{} .
2018 k\"undete auch Microsoft Discord Unterst\"utzung f\"ur ihre XBox Live Nutzer an und unterst\"utzte Discord mit einer Finanzierung
von 150 Millionen US-Dollar. Bewertet wurde Discord Inc nun auf etwa 2 Milliarden US-Dollar.

Aufgrund der Covid-19 Pandemie und den steigenden Benutzerzahlen stellte sich Discord auch f\"ur andere Zielgruppen ein.
So wurde der Fokus von Videospielen weggelenkt auf einen universellen Kommunikations-und Chat Client, um es f\"ur Branchen
wie Schule oder Arbeit ansprechender zu machen. Damit \"anderte das Unternehmen ihr Motto von \textit{Chat for Gamers} zu
\textit{Chat for Communities and Friends} und f\"uhrte Servervorlagen ein.

Heute hat Discord mehr als 140 Millionen monatlich aktive User, verwaltet etwa 13.5 Millionen aktive Server und
wird auf 15 Milliarden US-Dollar gewertet. (Alles Stand 2021) \autocite{}

\autocite{discord_wiki}


\subsubsection{Discord Einschr\"ankungen}

% https://netcord.site/new-discord-channel-creation-ratelimit/
% https://support.discord.com/hc/en-us/community/posts/360056762431-Increase-channel-limit
% https://support.discord.com/hc/en-us/community/posts/360032363631-Increase-the-Max-Number-of-Roles-per-Server

\subsection{Alternativen zu Discord}

\subsubsection{Slack}

Slack ist eine sehr ähnliche Alternative zu Discord, richtet sich jedoch mehr an Unternehmen als an Communities.
Dabei liegt der Fokus der Anwendung auch stärker auf dem Textchat als auf dem Sprachchat, wo sich Discord mehr fokusiert.
% TODO: add source

\subsection{Skype}


\subsection{TeamSpeak}

TeamSpeak macht es möglich einfach und ohne viel Rechenressourcen einen Server zum Austauschen aufzusetzen.
Auch wenn Textchats möglich sind, sind diese stärker eingeschränkt und klar nicht der Fokus der Anwendung.
% TODO: check this statement
Da Textchat der stärkere Fokus ist für den Austausch den STAIR anbieten will,
ist dies wohl keine geeignete Lösung.

\section{Ideen und Konzepte}

\todo{einf\"ugen}

\section{Methoden}

\todo{einf\"ugen}

\subsection{Technologien}

\subsubsection{C\# und .NET Framework}

\subsubsection{Mysql}

\todo{einf\"ugen}
% https://www.nuget.org/packages/linq2db.MySql/
% http://www.primaryobjects.com/2009/01/24/using-mysql-and-linq-to-sql-in-c-asp-net/

\section{Realisierung}

\subsection{Analyse der bestehenden Infrastruktur}
Als erster Schritt wird eine Analyse des bestehenden Bots und seiner Funktion im Discord durchgef\"uhrt.

\subsubsection{Aufbau Discord}
Der Discord Server "Stair" wird von Stair verwaltet und ist ein online Treffpunkt f\"ur alle Studenten des Departements Informatik.
Beim erstmaligen Eintretten in den Server hat man noch keine Berechtigung und es wird deshalb noch nichts, ausser dem Help Channel
angezeigt. Man bekommt vom Bot Stan, beim Eintretten auf den Server, eine Nachricht mit einer Anleitung. Darin ist beschrieben, wie
man sich authentifizieren kann. Bei der Authentifizierung wird vom Bot geschaut ob es sich um eine Studenten E-Mail Adresse handelt.
Wenn dem so ist, schickt der Bot einen 6-stelligen Random Code, den der Benutzer im Discord dem Bot zur\"uckschreiben muss.

Hat dieser Prozess funktioniert, ist man "eingeloggt" und bekommt vom Bot die Rolle Student zugeteilt. Damit hat man Zugriff auf
die verschiedenen Grundchannels, wie Gaming, Administration, General, Studying, etc.
In diesen Channels kann man sich mit seinen Mitstudierenden Sprachlich oder per Text \"uber alle Themen austauschen.
\newline

Der Stair Discord-Server bietet auch Informationen und Unterst\"utzung f\"ur alle Module an. Dabei hat jedes Modul einen eigenen Channel.
Dort k\"onnen sich Studierende gezielt \"uber ein Modul austauschen, Fragen stellen oder Informationen mitteilen.
Am Anfang wird gar nichts angezeigt. Man muss sich spezifisch f\"ur die Module registrieren um diese angezeigt zu bekommen. Dies verhindert

1. dass man unn\"otig Benachrichtigungen von Channels bekommt, die einen nicht interessieren.
2. dass sein Discord nicht \"uberf\"ullt wird mit 400 Modul-Channels.

Die Commands \textit{show <module>} und \textit{hide <module>} erlauben es, den Channel hinzuzuf\"ugen oder zu entfernen.
\newline

Neu gibt es auch das Konzept von verschiedenen H\"ausern bei Stair. Jeder Student wird dabei einem Haus zugeteilt. Die Zuteilung l\"auft
dabei \"uber das Sekretariat der Hochschule. Es wird geschaut, dass in jedem Studiengang in jedem Startsemester, die Studenten gleichm\"assig
auf die H\"auser verteilt werden.
Es gibt die H\"auser Blue, Purple, Red, Orange, Yellow und Grey. \"uber das Semester hinweg organisiert Stair verschiedene Events, bei denen
man Punkte f\"ur sein Haus sammeln kann. Am Ende jedes Jahres, also Ende Fr\"uhlingssemster, wird das Haus, welches am meisten Punkte
gesammelt hat, als Gewinner gelobt.

Der Discord-Server bietet die Platform um sein Haus zu mobilisieren oder die neuesten Ergebnisse den Studenten zu verk\"unden.
Pro Haus gibt einen Channel. Momentan werden die Studenten noch manuell zu ihren zugeh\"origen Channels hinzugef\"ugt.
\newline
\newline
\todo{Bild von Discord und Channels anzeigen}

\subsubsection{Funktionsweise Bot}
Der Bot ist mit der Programmiersprache C\# und dem .NET Framework geschrieben.
In der folgenden Tabelle ist eine \"Ubersicht der benutzen IDE, ihrer Spezifizierung und den eingesetzten Libraries.
\todo{Tabelle mit Spezifikationen vom System. (Versionen, Libraries)}
\newline


Das Programm, welches den Bot zur Verfügung stellt, ist nicht gross. Es beinhaltet im Groben einen Client Socket der
Events on Discord abfängt und einer E-Mail Klasse, die E-Mails an den Benutzer versenden kann.
Das Programm ist in zwei Packages aufgeteilt.

- StanBot.Core
- StanBot.Service

\todo{Bild Klassendiagramm Bot}

\subsubsection*{StanBot.Core}

\subsubsection*{StanBot.Service}


\subsubsection{Ausführung Bot}

Der Bot läuft auf einem Windows Server \#\#\#\# auf der Enterprise Lab Umgebung der HSLU. Die Konfiguration entspricht der eines
Windows Services und läuft dementsprechend durchgehend.
\newline
\todo{Bild Service Konfiguration}
\newline


\subsubsection{Geplante Änderungen}




\section{Evaluation und Validation}

\todo{schreiben}

\section{Ausblick}

\todo{schreiben}

\newpage

\section{Anh\"ange}

\todo{schreiben}

\section{Glossar}

% https://blog.aristolo.com/de/definition-begriffe-bachelorarbeit-masterarbeit-dissertation/
\todo{schreiben}

\section{Abbildungsverzeichnis}
\listoffigures

\section{Tabellenverzeichnis}
\listoftables

\section{Literaturverzeichnis}
\printbibliography

\end{document}
